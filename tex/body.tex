%%% TODO: In report, give small example where Single is needed

\usepackage[T1]{fontenc}
\usepackage{amsmath}
\usepackage{amssymb}
\usepackage[scaled=.82]{beramono}
\usepackage[scaled=.86]{helvet}
\usepackage{booktabs}
\usepackage{bussproofs}
\usepackage{cite}
\usepackage{mathptmx}
\usepackage{mathpartir}
\usepackage{stmaryrd}
\usepackage{subfigure}
\usepackage{textcomp}
\usepackage{url}
\usepackage[all]{xy}

\newcommand\beforefig{\rm}

\newcommand\DISC[1]{t \teq \const{C}_{#1}\negvthinspace\bigl(\const s^1_{#1}( t ), \ldots,\const s^{n_{#1}}_{#1}( t )\bigr)}
%\newcommand\DISC[1]{\const{d}_{#1}(t)}

%\newcommand\XXXL{DPLL}
\newcommand\XXXL{CDCL}

\newcommand\NOK[1]{\textcolor{red}{\textbf{#1}}}
\newcommand\OK[1]{#1}

%\newcommand\typ[1]{: #1}
\newcommand\typ[1]{^{\vthinspace #1}}

%\newcommand\PART[1]{.#1}
\newcommand\PART[1]{(#1)}

%\newcommand\AST{\ast}
\newcommand\AST{\infty} %%% matter of taste -- revert if you don't like

\newcommand\MU{\vvthinspace\mu\vvthinspace}

%%% not part of shared counter

\makeatletter
\def\spnXwtheorem#1#2#3#4{\@spynthm{#1}{#2}{#3}{#4}%
                         \@addtoreset{#1}{chapter}}%
\makeatother

\spnXwtheorem{examplex}{Example}{\itshape}{\rmfamily}

\newcommand\MIDRULE{
\\[-1pt] %%% TYPESETTING HACK
\midrule
%$\enatT$
\\[-11pt] %%% TYPESETTING HACK
}

\renewcommand\models{\vDash} %%% matter of taste

\def\thewordpaper{paper}
\newcommand\dotReportFootnote[1]{.}

\newcommand\FV{\mathrm{FV}}

\newcommand\betweenantes{\kern1.125em}

\newcommand\afterDot{\;} %%% Too little space by default after "Lemma Foo."

\begin{rep}
\def\thewordpaper{report}
\renewcommand\dotReportFootnote[1]{.\footnote{#1}}
\end{rep}

\newcommand\afterLdots{\kern.1em} %% TYPESETTING

% for "bussproofs" package
\EnableBpAbbreviations
\def\ScoreOverhang{1.5pt}
\def\proofSkipAmount{\vskip 0pt}
\def\defaultHypSeparation{\hskip0.75em}

\DeclareFontFamily{OT1}{pzc}{}
\DeclareFontShape{OT1}{pzc}{m}{it}{<-> s * [1.10] pzcmi7t}{}
\DeclareMathAlphabet{\mathcalx}{OT1}{pzc}{m}{it}

\let\labelitemi=\labelitemii %% CHEAT!

\newcommand\cpp{C\nobreak\raisebox{.05ex}{+}\nobreak\raisebox{.05ex}{+}}

\newcommand\iPrime{i\vthinspace'\negvthinspace}
\newcommand\jPrime{j\vthinspace'\negvthinspace}

\newcommand\Sig{\mathrm{\Sigma}}

\newcommand\keyw[1]{\textbf{#1}}
\newcommand\const[1]{\textsf{#1}}
\newcommand\ty[1]{\textit{#1}}

%%% I don't like using \qed or anything too similar (e.g. \Box) for
%%% anything else than "quid erat demonstrandum". I'm open to other symbols,
%%% e.g. a filled black square. But if the example are short enough, which
%%% they currently are, we could live with nothing.
%%%
\newcommand\xend{{\hfill$\scriptstyle\blacksquare$}}
%\newcommand\xend{{\qed}}
%\newcommand\xend{{}}

%\renewcommand{\vec}[1]{\mathbf{#1}}
\renewcommand{\vec}[1]{\bar #1}
%\newcommand{\Ec}{\mathsf{E}}
\newcommand{\Ec}{E}
%\newcommand{\Ec}{\mathcalx{E}}
%\newcommand{\Fc}{\mathsf{F}}
\newcommand{\Fc}{F}
%\newcommand{\Fc}{\mathcalx{F}}
%\newcommand{\Gc}{\mathsf{G}}
%\newcommand{\Ac}{\mathsf{A}}
%\newcommand{\Dc}{\mathsf{D}}
%\newcommand{\Rc}{\mathsf{R}}
\newcommand{\vrange}{\mathsf{range}}
\newcommand{\vdom}{\mathsf{dom}}
\newcommand{\tEc}{\Terms(\Ec)}
\newcommand{\tcEc}{\Terms^\AST(\Ec)}
\newcommand{\rn}[1]{\textsf{#1}}
\newcommand{\teq}{\approx}
\newcommand{\tneq}{\not\teq}
\newcommand{\rem}[1]{\textcolor{red}{[#1]}}
%\newcommand{\ror}{\parallel}
%\newcommand{\ror}{{\&}}
%\newcommand{\ROR}{\kern.45em \ror \kern.45em}
\newcommand\ROR{\betweenantes}
\newcommand{\tpath}[2]{\mathcalx{P}_{ #2 \rightarrow \_ }( #1 )}
\newcommand{\ttpath}[3]{\mathcalx{P}_{ #2 \rightarrow #3 }( #1 )}

%\newcommand{\expand}[2]{#1\llbracket #2 \rrbracket}
%%% I prefer a notation that puts more emphasis on #2 (which is, in essence,
%%% the object that is returned) and less on #1
%\newcommand{\expand}[2]{\llbracket #2 \rrbracket_{#1}}
%%% And one that doesn't look so semantic
%\newcommand{\expand}[2]{\lceil #2 \rceil_{#1}}
\newcommand{\expand}[2]{\langle \smash{#2} \rangle_{#1}}

\newcommand{\interp}[2]{#1(#2)}
%\newcommand{\interp}[2]{{#2}^{#1}}

\newcommand{\dpath}[3]{\delta^{#1}_{#2}( #3 )}
\newcommand{\ec}[1]{[#1]}
\newcommand{\JJJJ}{\mathcal{\!J\!}}
\newcommand{\J}{\mathcalx{J}}
\newcommand{\Val}{\mathcalx{A}\vvthinspace}
\newcommand{\ValC}{\smash{\mathcalx{A}^\star}}
\newcommand{\Var}{\mathcalx{V}}
%\newcommand{\Varec}[1]{\Var{\ec{#1}}}
\newcommand{\Varec}[1]{\vvthinspace\widetilde{#1}\vvthinspace}
\newcommand{\VAREC}[1]{\widetilde{\vphantom{\scriptstyle x^i}\smash{#1}}}
%\newcommand{\nf}[1]{{{#1}{\downarrow}}}
\newcommand{\nf}[1]{\lfloor#1\rfloor}
\newcommand{\aequiv}{\mathrel{=_\alpha}}
\newcommand{\vsim}{\aequiv}
\newcommand{\vsimv}[1]{\mathrel{=^{#1}_\alpha}}
\newcommand{\tpos}[2]{#1\!\mid_{#2}}
%\newcommand{\muvar}{\mathcalx{Var}}
\newcommand{\muvar}{\mathrm{Var}} %% cf. FV

\newcommand\SSS{\mathit{S}}
\newcommand\SSSS[1]{\mathit{S}^{\,#1}}

% change?
\newcommand{\thO}{T_{\mathrm{o}}}
%\newcommand{\thD}{T_{\!\mathrm{cd}}}
%%% the italicized "T" makes it look as though it were a variable;
%%% and the subscript means we have two layers of subscripts in things like
%%% \model_\thD (although we could drop the "T" and keep only the subscript
%%% like you did in the string paper).
%%% How about the following? I'm pretty sure \mathcalx has been used for theories
%%% before by other authors.
\newcommand{\thD}{\mathcalx{DC}}

\newcommand\BAD{\mathcalx{V}}
\newcommand\Terms{\mathcalx{T}}
\newcommand\Types{\mathcalx{Y}}
\newcommand\Funcs{\mathcalx{F}}

\newcommand\Data{\Types_{\mathrm{dt}}}
\newcommand\Codata{\Types_{\mathrm{codt}}}
\newcommand\Nondata{\Types_{\mathrm{ord}}}

\newcommand\Ctr{\Funcs_{\smash{\mathrm{ctr}}}}
\newcommand\Sel{\Funcs_{\smash{\mathrm{sel}}}}
%\newcommand\Plainfuncs{\Funcs_{\mathrm{other}}}

\newcommand\vvthinspace{\kern+0.041667em}
\newcommand\vthinspace{\kern+0.083333em}
\newcommand\negvthinspace{\kern-0.083333em}
\newcommand\negvvthinspace{\kern-0.041667em}

%%% For final version as well?
\usepackage[
   a4paper,
   pdftex,
   pdftitle={A Decision Procedure for (Co)datatypes in SMT Solvers},
   pdfauthor={Andrew Reynolds and Jasmin Christian Blanchette},
   pdfkeywords={},
   pdfborder={0 0 0},
   draft=false,
   bookmarksnumbered,
   bookmarks,
   bookmarksdepth=2,
   bookmarksopenlevel=2,
   bookmarksopen]{hyperref}

\urlstyle{ttstyle}

\global\def\figurename{Figure}
\global\def\figuresname{Figures}

\DeclareSymbolFont{letters}{OML}{txmi}{m}{it}

%%% REMOVE BEFORE SUBMITTING ABSOLUTELY FINAL VERSION
\makeatletter
\ps@myheadings
\makeatother

\include{defs}

\hyphenation{data-type data-types co-data-type co-data-types isa-belle sledge-hammer
non-redun-dant counter-examples counter-example mono-morph-ize mono-morph-ized}


\begin{document}

\title{A Decision Procedure for (Co)datatypes in SMT Solvers}

\author {Andrew Reynolds\inst{1} \and Jasmin Christian Blanchette\inst{2,3}}
\authorrunning {A. Reynolds \and J. C. Blanchette}
\institute{
\'Ecole Polytechnique F\'ed\'erale de Lausanne (EPFL), Switzerland
\and
Inria Nancy \& LORIA, Villers-l\`es-Nancy, France
\and
Max-Planck-Institut f\"ur Informatik, Saarbr\"ucken, Germany
}

\maketitle

\vskip3\smallskipamount

\begin{nomemoriam}
\noindent\hfill\emph{\phantom{In memoriam Morgan Deters 1979--2015}}

\vspace*{-6\smallskipamount}
\end{nomemoriam}
\begin{memoriam}
\noindent\hfill\emph{\relax{In memoriam Morgan Deters 1979--2015}}
\end{memoriam}

\begin{abstract}
%Codatatypes naturally capture potentially infinite data structures and
%processes.
We present a decision procedure that combines reasoning about
datatypes and codatatypes. The dual of the acyclicity rule for datatypes is a
uniqueness rule that identifies observationally equal codatatype values,
including cyclic values. The procedure decides universal
problems and is composable via the Nelson--Oppen method. It has been
implemented in CVC4, a state-of-the-art SMT solver. An evaluation based on
problems generated from theories developed with Isabelle demonstrates the
potential of the procedure.
\end{abstract}

\begin{nomemoriam}
\vskip3\smallskipamount
\end{nomemoriam}

%% The institutions above shouldn't count as footnotes
\setcounter{footnote}{0}

\section{Introduction}
\label{sec:introduction}

Freely generated algebraic datatypes are ubiquitous in functional programs and
formal specifications. They are especially
useful to represent finite data structures in computer science applications but
also arise in formalized mathematics.
They can be implemented efficiently and enjoy
properties that can be exploited in automated reasoners.
%
%However, because datatype values correspond to finite ground terms, they
%are generally not adequate to represent infinite objects.
%For example, the datatype of natural numbers
%constructed by $\const{Z} : \ty{nat}$ and $\const{S} : \ty{nat} \to \ty{nat}$
%only allow values of the form $\const{S}(\ldots(\const{S}(\const{Z}))\ldots)$.

To represent infinite objects, % such as $\const{S}(\const{S}(\const{S}(\ldots))$,
a natural choice is to turn to coalgebraic datatypes, or \emph{codatatypes},
the non-well-founded dual of algebraic \emph{datatypes.}
%
Despite their reputation for being esoteric, codatatypes have a
role to play in computer science. The verified C compiler CompCert
\cite{leroy-2009}, the verified Java compiler Jinja\-Threads
\cite{lochbihler-2010-jinja}, and the formalized Java memory model
\cite{lochbihler-2014-jmm} all depend on codatatypes to capture infinite
processes.

Codatatypes are freely generated by their constructors, but in contrast with datatypes,
infinit\-e constructor terms are also legitimate values for codatatypes
(Section~\ref{sec:co-datatypes}). Intuitively, the
values of a codatatype consist of all well-typed finite and infinite ground
constructor
terms, and only those. As a simple example, the coalgebraic specification
%
\[\keyw{codatatype}~\,\ty{enat} \,=\, \const{Z} \,\mid\, \const{S}(\ty{enat})\]
%
%(using an ML-like syntax)
{\looseness=-1
introduces a type that
models the natural numbers $\const{Z}$, $\const{S}(\const{Z})$, $\const{S}(\const{S}(\const{Z}))$, $\ldots$\afterLdots{},
in Peano notation but extended with an
infinite value $\infty = \const{S}(\const{S}(\const{S}(\ldots)))$.
The equation $\const{S}(\infty) \teq \infty$ holds as expected,
because both sides expand to the infinite term
$\const{S}(\const{S}(\const{S}(\ldots)))$, which uniquely identifies
%the value~
$\infty$.
Compared with the conventional %algebraic
definition
\begin{conf}$\vthinspace \keyw{datatype}~\,\ty{enat} \,=\, \const{Z} \,\mid\, \const{S}(\ty{enat}) \,\mid\, \const{Infty}$\vvthinspace, \end{conf}%
\begin{rep}\[\vthinspace \keyw{datatype}~\,\ty{enat} \,=\, \const{Z} \,\mid\, \const{S}(\ty{enat}) \,\mid\, \const{Infty}\]\end{rep}%
%with a dedicated constructor for representing infinity, the
the codatatype avoids
one case by unifying the infinite and finite nonzero cases.

}

{\looseness=-1
Datatypes and codatatypes are an integral part of many proof assistants,
including Agda, Coq, Isabelle, Matita, and PVS. In recent years, datatypes
have emerged in a few automatic theorem provers as well. The SMT-LIB
%\cite{barrett-et-al-2010}
format, implemented by most SMT
%(satisfiability modulo theories)
solvers, has been extended with a syntax for datatypes.
In this \thewordpaper, we introduce a
unified \hfill decision \hfill procedure \hfill for \hfill universal \hfill problems \hfill involving \hfill datatypes \hfill and \hfill codatatypes%
\begin{conf}\goodbreak\noindent\end{conf}%  %%% TYPESETTING BREAK
\begin{rep} \end{rep}%
in combination (Section~\ref{sec:the-decision-procedure}).
The procedure is described abstractly as a %tableau-like
calculus and is composable via the Nelson--Oppen method \cite{nelson-oppen-1979}.
It generalizes the procedure by Barrett et al.\ \cite{barrett-et-al-2007},
which covers only datatypes.
%To our knowledge, our procedure is the first of its kind for the theory of
%codatatypes.
\begin{conf}%
Detailed proofs %of completeness
are included in a %technical
report
\cite{our-report}.
\end{conf}%

}

Datatypes and codatatypes share many properties, so it makes sense
to consider them together. There are, however, at least three important
differences.
%
First, \emph{codatatypes need not be well-founded.}
For example, the type

\begin{conf}
\kern-2pt %%% TYPESETTING: Crude hack
\end{conf}

\vskip\abovedisplayskip
\noindent\centerline{$\keyw{codatatype}~\;\ty{stream}_{\,\tau} \,=\, \const{SCons}(\tau,\: \ty{stream}_{\,\tau})$}
\vskip\belowdisplayskip

\begin{conf}
\kern-2pt %%% TYPESETTING: Crude hack
\end{conf}

\noindent
of infinite sequences or streams over an element type $\tau$ is
allowed\begin{conf}, even though it has no base case\end{conf}\begin{rep}; the
corresponding datatype would be rejected as non-well-founded
\cite{blanchette-et-al-2015-esop}\end{rep}.
%
Second, \emph{a uniqueness rule takes the place of the acyclicity rule of datatypes.}
Cyclic constraints such as
$x \teq \const{S}(x)$ %, where $\const{C}$ is a constructor,
are unsatisfiable for datatypes,
thanks to an acyclicity rule,
but satisfiable for codatatypes. 
For the latter, a uniqueness rule ensures that two values
having the same infinite expansion must be equal; from $x
\teq \const{S}(x)$ and
$y \teq \const{S}(y)$, it deduces $x \teq y$.
These two rules are needed to ensure completeness
(solution soundness) on universal problems. %and the absence of spurious models.
They cannot be expressed as finite axiomatizations, so they naturally belong in
a decision procedure.
%    * in particular, acyclicity and uniqueness are necessary for some proofs
%%      (and cannot be axiomatized finitely)
%    * and for model finding, without them we quickly get spurious models
%    * explain how finite model finding works
%(For the other (co)datatype properties---the injectivity, distinctness, and
%exhaustiveness of constructors and the selector laws---it is
%widely recognized that decision procedures can be more efficient than
%axiomatizations.)
%axiom.
%
Third, \emph{it must be possible to express cyclic }(\emph{regular}) \emph{values as closed terms and
to enumerate them.} This is necessary both for finite model finding \cite{reynolds-et-al-2013}
%modulo theories
and for theory combinations. The $\mu$-binder notation associates a name with
a subterm; it is used to represent cyclic values in the generated
models\begin{rep} and in the metatheory\end{rep}. For example,
the $\mu$-term $\const{SCons}(1,\: \MU s.\; \const{SCons}(0,\: \const{SCons}(9,\: s)))$
stands for the lasso-shaped sequence $1, 0, 9, 0, 9, 0, 9,
\ldots$\afterLdots.

%%% 1090909 is a prime number

Our procedure is implemented in the SMT solver CVC4 \cite{barrett-et-al-2011} as a combination
of rewriting and a theory solver (Section~\ref{sec:implementation-as-a-theory-solver-in-cvc4}).
It consists of about 2000 lines of \cpp{} code, %among which \OK{1600} are
most of which are shared between datatypes and codatatypes. The code is
integrated in the development version of the solver and is expected to be part
of the CVC4~1.5 release.
%
An evaluation on %hand-crafted examples and on
problems generated from Isabelle theories using the Sledgehammer tool \cite{blanchette-et-al-2013-smt}
demonstrates the usefulness of the approach (Section~\ref{sec:experimental-results}).

%  * useful both for proving and for model finding

%\ref{sec:examples}

%  * benchmarking is often an issue -- esp. codatatypes
%
%Polymorphic types, nested (co)recursion, and datatype--codatatype mixtures fall

%  * setting: FOL
%    * core procedure is restricted to ground
%      * but theory solver cooperates

%  * codatatypes were added later, motivated by the use of SMT solvers as
%    backends to proof assistants (more specifically, CVC4 to Isabelle/HOL)

%  * setting :
%    * universal formulas
%    * many-sorted logic
%   * mutually (co)recursive types with constructors, selectors, and
%      discriminators

%  * codatatypes: from a theoretical and implementational point of view, like
%      datatypes but:
%    * infinite values (infinitely many nested constructors)
%    * codatatypes are never empty (e.g. finite streams are rejected)

%\[
%      \keyw{codatatype}\; \,\ty{llist} \,=\, \const{LNil} \,\mid\, \const{LCons}(\ty{int},\: \ty{llist})
%\]

%  * consider a natural fragment---datatypes as supported in modern SMT solvers
%    and the SMT-LIB 2 standard, and codatatypes as their duals
%    * mutual recursion, but no polymorphism, nested recursion
%  * integrated with Nelson--Oppen

% * SMT-Lib

%  * one implication is that if $m$ equals $ES(n)$ and $n$ equals $ES(m)$, necessarily
%    $m$ and $n$ must be equal.

%  * perhaps the most commonly used codatatype is that of lazy lists or sequences.
%    using a syntax similar to Standard ML, Haskell, or SMT-LIB

%  * less briefly: codatatypes and why they are useful
%    * in Agda, Coq, Matita -- recently also in Isabelle/HOL \cite{nipkow-et-al-2002}
%    * but not in SMT-LIB 2 \cite{barrett-et-al-2010}

\begin{rep}\paragraph{Related Work.} Barrett\end{rep}%
\begin{conf}Barrett\end{conf}
et al.\ \cite{barrett-et-al-2007} provide a good account of
related work on datatypes as of 2007, in addition to describing their
implementation in CVC3.
Since then, datatypes have been added not only to CVC4 (a complete rewrite of CVC3)
but also to the SMT solver
Z3 \cite{de-moura-bjoerner-2008} \begin{rep}in unpublished work by Leonardo de Moura \end{rep}and \begin{rep}to \end{rep}a SPASS-like %prototypical
superposition prover \begin{rep}by Daniel Wand\end{rep} \cite{wand-2014}.
Closely related are the automatic structural
% cruanes-201x,wand-weidenbach-201x,  -- both were rejected
induction in both kinds of provers \cite{kersani-peltier-2013,reynolds-kuncak-2015},
the (co)datatype and (co)induction support in Dafny \cite{leino-moskal-2014},
and the (semi-)decision procedures for datatypes
implemented in Leon \cite{suter-et-al-2011} and RADA \cite{pham-et-al-2013}.
%
Datatypes are supported by the higher-order model finder Refute
\cite{weber-2008}. Its successor, Nitpick \cite{blanchette-nipkow-2010}, can
also generate models involving cyclic codatatype values.
Cyclic values have been studied extensively under the heading of
regular or rational trees---see Carayol and Morvan
\cite{carayol-morvan-2006} and Djellou et al.\ \cite{djelloul-et-al-2008}
for recent work.
The $\mu$-notation is inspired by the $\mu$-calculus
\begin{conf}\cite{kozen-1983}.\end{conf}
\begin{rep}\cite[etc.]{kozen-1983,endrullis-et-al-2011}.\end{rep}

%, notably in the context of constraint logic programming

%  * Ro\c{s}u --- equality of streams is...
%     http://dl.acm.org/citation.cfm?doid=1159803.1159827

%     * point to Barrett et al. for SMT datatypes
%       * about their own work, they say: "our focus is on generality and
%         efficiency rather than immediacy of implementation"
%       * (deal more directly with finite sorts than Barrett et al., Section 6.1)
%     * additional ones since then (e.g. strings?)
%     * anything about codatatypes?
%       * proof assistants like Agda, Coq, etc. have them
%       * Dafny, CoALP
%       * also a lot of theoretical research, some of which is loosely connected,
%         e.g. decision procedure for corecursive functions (Henning in Nijmegen)
%
%     * ODDITY: Oppen 1980: single-constructor, recursive -- infinite values?

\begin{conf}
\kern-3pt %%% TYPESETTING: Crude hack
\end{conf}

\paragraph{Conventions.}
Our setting is a monomorphic (or many-sorted) first-order logic.
A signature $\Sig = (\Types, \Funcs)$ consists of a set of types $\Types$ and a
set of function symbols $\Funcs$. Types are atomic sorts and interpreted as
nonempty domains. The set~$\Types$ must contain a
\begin{rep}distinguished \end{rep}type \ty{bool} interpreted as the set of truth
values. %, and may contain other interpreted types (e.g., \ty{int}, \ty{real}).
\begin{rep}
The metavariables $\delta,\:\varepsilon$ range over (co)datatypes,
whereas $\tau,\:\upsilon$ range over arbitrary types.
When applied to terms, the symbol $=$ denotes syntactic equality.

Function symbols are written in a sans-serif font (e.g., $\const{f}$, $\const{g}$) to
distinguish them from variables (e.g., $x$, $y$).
\end{rep}
Names starting with an uppercase letter \begin{rep}(e.g.,
$\const{S}$) \end{rep}are reserved for constructors. With each function symbol \const{f}
is associated a list of argument types $\tau_1,\ldots,\tau_n$ (for $n \ge 0$)
and a return type $\tau$, written
$\const{f} : \tau_1 \times \cdots \times \tau_n \to \tau$\begin{rep},
which collapses to $\const{f} : \tau$ if $n = 0$\end{rep}.
\begin{rep}%
Functions invocations $\const{f}(t_1,\ldots,t_n)$
apply the $n$-ary function symbol
\const{f} to $n$ well-typed arguments $t_1 :\nobreak \tau_1$, \ldots, $t_n :
\tau_n$. Nullary function symbols, called constants, can appear without
parentheses in terms.
\end{rep}
The set $\Funcs$ must at least contain
$\const{true},\, \const{false} : \ty{bool}$, interpreted as truth values.
The only predicate is equality ($\teq$)\begin{rep} and belongs to the logical symbols;
other predicates can be represented as functions to $\ty{bool}$,
with $\const{p}(\ldots)$ abbreviating $\const{p}(\ldots) \teq \const{true}$\end{rep}.
The notation $t\typ{\tau}$ stands for a term $t$ of type~$\tau$.
\begin{rep}The operator $\bigwedge_{\,i}\, \varphi_i$ abbreviates a conjunction
$\varphi_1 \mathrel\land \cdots \mathrel\land \varphi_n$.
Finally, $\bar x$ abbreviates a list or tuple $x_1,\ldots,x_n$.\end{rep}

%* although nothing prevents composing the decision procedure with theories
%  providing polymorphic types (parametric sorts), such as for arrays (e.g., $\ty{array}(\alpha,\beta)$)

\section{%The Theory of
(Co)datatypes}
\label{sec:co-datatypes}

%    (terminology: freely-generated, inductive, algebraic, ..., sometimes
%    with different meanings; we'll clarify below what we use)

{\looseness=-1
We fix a signature $\Sig = (\Types, \Funcs)$. The types are partitioned into
$\Types = \Data \mathrel{\uplus} \Codata \mathrel{\uplus} \Nondata$, where $\Data$ are the
\emph{datatypes}, $\Codata$ are the \emph{codatatypes}, and $\Nondata$ are the %remaining
\emph{ordinary types} (which can be interpreted or not). The function symbols are
partitioned into $\Funcs = \Ctr \mathrel{\uplus} \Sel$, where $\Ctr$ are the
\emph{constructors} and $\Sel$ are the \emph{selectors}. There is no need to
consider further function symbols because they can be abstracted away as
variables when combining theories. $\mathrm\Sigma$-terms are standard first-order terms
over $\mathrm\Sigma$,
without $\mu$-binders.
%(Section~\ref{sec:implementation-as-a-theory-solver-in-cvc4}).
%Exceptionally, it is convenient to use numeric constants ($0$, $1$, \ldots)\ in
%examples.

}

%\paragraph{Specifications.}
In an SMT problem, the signature is typically given by specifying first the
uninterpreted %ordinary
types in any order, then the (co)datatypes with their constructors
and selectors in groups of mutually (co)recursive groups of (co)datatypes, and
finally any other function symbols.
%
Each (co)datatype specification consists of $\ell$~mutually recursive types that are
either all datatypes or all codatatypes. \begin{rep}Polymorphic types, nested
(co)recursion,\end{rep}\begin{conf}Nested (co)recursion\end{conf} and datatype--codatatype mixtures
fall outside this fragment%
\dotReportFootnote{%
In principle, rank-1 (top-level) polymorphism \cite{blanchette-paskevich-2013}
should not raise any special difficulties. Nesting datatypes inside datatypes,
and likewise for codatatypes, can be reduced to the mutual case
\cite{gunter-1993-not}. So the only genuinely interesting cases missing are
mixed nested (co)recursion as well as
(co)recursion through a non-(co)datatype (both of which make sense
\cite{blanchette-et-al-2014-codata}).}
\begin{rep}We allow ourselves some notational parameterization
% at the metalevel
through subscripts---for example, $\ty{stream}_{\,\tau}$ denotes a
family of ground types including
$\ty{stream}_{\,\ty{int}}$, $\ty{stream}_{\,\ty{bool}}$,
and \smash{$\ty{stream}_{\,\ty{stream}_{\,\ty{real}}}$}.\end{rep}

\newcommand\elll{\kern.0515ex \ell\kern.0515ex}
\newcommand\elllx{\kern.0515ex \ell\kern.0515ex}

Each (co)datatype $\delta$ is equipped with
$m \ge 1$ constructors, and each constructor for $\delta$ takes zero or more
arguments and returns a $\delta$ value. The argument types must be either
ordinary, among the already known (co)datatypes, or among the (co)datatypes
being introduced.
%
To every argument corresponds a selector. The names for the (co)data\-types, the
constructors, and the selectors must be
\begin{conf}fresh\end{conf}\begin{rep}distinct and different from
existing names\end{rep}%
\dotReportFootnote{It can be convenient to specify the same selector
for several constructors associated with the same (co)data\-type,
as long as the argument types coincide. % \cite[Section~3]{blanchette-et-al-2014-codata}.
However, this is disallowed by CVC4 and Z3, so we do not consider it here.}
Schematically:
%
\[
\begin{aligned}[t]
\!(\keyw{co})\keyw{datatype}\;\,
  \delta_1 & {}=\, \smash{\const{C}_{11\!}(\bigl[\const{s}_{11\!}^1{:}\bigr]\vthinspace \tau_{11\!}^1, \ldots, \bigl[\const{s}_{11\!}^{n_{11\!}}{:}\bigr]\vthinspace \tau_{11\!}^{n_{11\!}})} \mid \cdots \mid \smash{\const{C}_{1m_1\!}(\ldots)} \\[-.5\jot]
   \smash{\vdots\,\,\,} \\[-1.5\jot]
  \keyw{and}\; \,\delta_{\elllx} & =\, \smash{\const{C}_{\elll 1\!}(%\bigl[\const{s}_{\elll 1\!}^1{:}\bigr]\vthinspace \tau_{\elll 1\!}^1, \ldots, \bigl[\const{s}_{\elll 1\!}^{n_{\elll 1\!}}{:}\bigr]\vthinspace \tau_{\elll 1\!}^{n_{\elll 1\!}}
  \ldots)} \mid \cdots \mid \smash{\const{C}_{\elll m_{\elllx}\!}(\ldots)}
\end{aligned}
\]
%
with
$\smash{\const{C}_{i\negvthinspace j} : \tau_{i\negvthinspace j}^1\times\cdots\times\tau_{i\negvthinspace j}^{\,n_{\smash{i\negvthinspace j}}} \to \delta_i}$
and $\smash{\const{s}_{i\negvthinspace j}^{\,k} : \delta_i \to \tau_{i\negvthinspace j}^{\,k}}$.
Defaults are assumed for the selector names if they are omitted.
The $\delta$ constructors and selectors are denoted by $\Ctr^\delta$ and
$\Sel^\delta$.
%
For types with several constructors, it is useful to provide discriminators
$\const{d}_{i\negvthinspace j} : \delta_i \to \ty{bool}$. Instead of extending $\Funcs$,
we let $\const{d}_{i\negvthinspace j}(t)$
be an abbreviation for
$t \teq \const{C}_{i\negvthinspace j}\bigl(\const{s}_{i\negvthinspace j}^1(t), \ldots, \const{s}_{i\negvthinspace j}^{\,n_{\smash{{i\negvthinspace j}}}}(t)\bigr)$.
%This will simplify the presentation. % of the decision procedure.

\begin{rep}%
Here are a few examples of legal specifications of (co)datatype families:
\[\begin{aligned}[t]
      \keyw{codatatype}\; \,\ty{llist}_{\,\tau} & \,=\, \const{LNil} \,\mid\, \const{LCons}(%\const{lhead}{:}\;
      \tau,\: %\const{ltail}{:}\;
      \ty{llist}_{\,\tau}) \\%[-.5\jot]
      \keyw{datatype}\;\, \ty{tree}_{\,\tau} & \,=\, \const{Node}(\tau,\:\, \ty{forest}_{\,\tau}) \\[-\jot]
      \keyw{and}\;\,\, \ty{forest}_{\,\tau} & \,=\, \const{FNil} \,\mid\, \const{FCons}(\ty{tree}_{\,\tau},\:\, \ty{forest}_{\,\tau})
\end{aligned}
\]

Because all types must be inhabited (nonempty), a datatype specification is
admissible only if a ground constructor term can be exhibited.
This rules out non-well-founded specifications such as
\[\keyw{datatype}\;\, \ty{fstream}_{\,\tau} \,=\, \const{FSCons}(\tau,\:\, \ty{fstream}_{\,\tau})\]
For codatatypes, no admissibility check is necessary because there is always a term,
finite or infinite, that witnesses nonemptiness \cite{blanchette-et-al-2015-esop}.
\end{rep}

A type $\delta$ depends on another type $\varepsilon$ if $\varepsilon$ is the
type of an argument to one of $\delta$'s constructors. Semantically, a set of
types is \emph{mutually} (\emph{co})\emph{recursive} if and only if the
associated dependency graph is strongly connected.
A type is (\emph{co})\emph{recursive} if it belongs to such a set of types.
\begin{rep}%
Types can be declared together in a mutual fashion even if they are not actually
(co)recursive. The semantic notion is more precise and is the one that interests
us.
\end{rep}%
%
Non(co)recursive type specifications such~as
\begin{conf}%
\vthinspace $\keyw{datatype}~\vthinspace\ty{option}_{\,\tau} \,=\, \const{None} \,\mid\, \const{Some}(\tau)$
\end{conf}%
\begin{rep}%
\[\begin{aligned}[t]
      \keyw{datatype}~\vthinspace\ty{option}_{\,\tau\!} & \,=\, \const{None} \,\mid\, \const{Some}(\tau) \\[-.5\jot]
      \keyw{codatatype}~\vthinspace\ty{complex} & \,=\, \const{Complex}(\const{re}{:}\; \ty{real},\; \const{im}{:}\;\ty{real})
\end{aligned}
\]
\end{rep}%
are permitted.
\begin{rep}At the semantic level, it makes no difference whether
such types are introduced as datatypes or as codatatypes.
\end{rep}%
%Without loss of generality, we consider that these types are datatypes
%instead of codatatypes.

%Given the specification
%
%\[\keyw{codatatype}\;\, \ty{stream} \,=\, \const{SCons}(\ty{int},\:\ty{stream})\]
%
%the infinite value $\const{SCons}(0, \const{SCons}(0, \ldots))$ witnesses
%nonemptiness.

One way to characterize datatypes %semantically
is as the initial model of the
selector--constructor equations \cite{barrett-et-al-2007}.
\begin{rep}
A drawback of this approach is that it does
not naturally account for selectors applied to wrong constructors. Barrett et
al.\ address this by parameterizing the construction by default values, but
this gives rise to spurious equalities between unrelated terms---e.g.,
$\const{s}_1^1(\const{C}_2) \teq \const{s}_1^1(\const{C}_3)$. This flaw
could be corrected, but the added complexity seems to suggest that selectors
are better characterized axiomatically.

\end{rep}
A related semantic view of datatypes is as initial algebras. Codatatypes are
then defined dually as final coalgebras \cite{rutten-2000}. The datatypes are
generated by their constructors, whereas the codatatypes are viewed through
their selectors.
%By uniformly focusing on the constructors, the
%axiomatic approach emphasizes the commonality between datatypes and
%codatatypes, while sacrificing a theoretically fruitful notion of duality.

Datatypes and codatatypes share many basic properties\begin{rep}. All properties
below are implicitly universally quantified and range over all $i$, $j$,
$\jPrime$, and $k$ within bounds\end{rep}:
% and to all possible splits of the
%$n$-ary constructor $\const{C}_{i\negvthinspace j}$'s argument list into $\bar x,
%y, \bar z$:
%
\[
\begin{aligned}[t]
\text{Distinctness:}\quad
  & %\forall \bar x, \bar y.\;\,
    \smash{\const{C}_{i\negvthinspace j}(\bar x) \tneq \const{C}_{i\negvthinspace \jPrime}(\bar y) \quad\text{if $j \not= \jPrime$}}
  \\[-0.5\jot]
\text{Injectivity:}\quad
  & %\forall x_1,\ldots,x_{n_{i\negvthinspace j}} y.\;\,
    \smash{\const{C}_{i\negvthinspace j}(x_1,\ldots,x_{n_{i\negvthinspace j}}) \teq \const{C}_{i\negvthinspace j}(y_1,\ldots,y_{n_{i\negvthinspace j}}) \longrightarrow x_k \teq y_k}
  \\[-0.5\jot]
\text{Exhaustiveness:}\quad
  & \smash{\const{d}_{i1}(x) \mathrel\lor \cdots \mathrel\lor \const{d}_{im_i}(x)}
  \\[-0.5\jot]
\text{Selection:}\quad
  & \smash{\const{s}_{i\negvthinspace j}^{\,k}(\const{C}_{i\negvthinspace j}(x_1,\ldots,x_{n_{i\negvthinspace j}})) \teq x_k}
\end{aligned}
\]
%
\begin{rep}%
Expressed in the algebraic jargon, exhaustiveness helps ensure that ``no
junk'' exists, whereas distinctness and injectivity guarantee that ``no
confusion'' can arise.
The result of selectors applied to the wrong
constructor is left completely unspecified.
\end{rep}%
%
%  * how to deal with "wrong" selectors, e.g.
%        hd(nil1) = hd(nil2)?
%      * leave them unspecified; hence hd(nil1) = hd(nil2) in some models,
%        and not in other models
%
Datatypes are additionally characterized by an induction principle\begin{rep}:
%
\[
\begin{aligned}[t]
\text{Induction:}\quad
\AXC{\strut$\bigwedge_{\,i,j}\, \forall x_1 \ldots \vthinspace x_{n_{i\negvthinspace j}}.\; \bigl(\bigwedge_{\,k}\, \mathit{IH}_{i\negvthinspace j}^k[x_k]\bigr) \longrightarrow P_i[\const C_{i\negvthinspace j}(x_1,\ldots,x_{n_{ij}})]$}
\UIC{\strut$\bigwedge_{\,i}\, P_i[v_i]$}
\DP
\end{aligned}
\]
where the induction hypothesis $\mathit{IH}_{i\negvthinspace j}^k(x)$
denotes either $P_{\iPrime}(x)$ if there exists some $\iPrime$ such that
the formula is type-correct or else $\top$\end{rep}.
The principle ensures that the interpretation of datatypes is standard.
For the natural numbers constructed from $\const{Z}$ and $\const{S}$,
induction prohibits models that contain cyclic values---e.g.,
an $n$ such that $n \teq \const{S}(n)$---or even infinite acyclic values
$\const{S}(\const{S}(\ldots))$.

\pagebreak[2]

For codatatypes, the dual notion is called coinduction. \begin{rep}It depends on
witnesses $R_i$ that are required to be bisimulations:
%
\[
\begin{aligned}[t]
\text{Coinduction:}\quad
\AXC{\strut$\begin{gathered}\textstyle \smash{\bigwedge_{\,i}\, R_i[v_i, w_i]} \\[-\jot]\textstyle
\bigwedge_{\,i}\, \forall v\; w.\;\, R_i[v, w] \longrightarrow
  \bigwedge_{\,j}\, \const{d}_{\negvthinspace j}(v) \teq \const{d}_{\negvthinspace j}(w)
  \mathrel\land
  \const{d}_{\negvthinspace j}(v) \longrightarrow \bigwedge_{\,k}\, \const{s}_{i\negvthinspace j}^{\,k}(v) \sim \const{s}_{i\negvthinspace j}^{\,k}(w)
\end{gathered}$}
\UIC{\strut$\bigwedge_{\,i}\, v_i \teq w_i$}
\DP
\end{aligned}
\]
where $x \sim y$ denotes either $R_{\iPrime}[x, y]$ if there exists some
$\iPrime$ such that the formula is type-correct or $x \teq y$ otherwise.
\end{rep}%
This axiom encodes a form of extensionality:\ Two values that yield the same
observations must be equal, where the observations are made through
selectors and discriminators.
%
\begin{rep}\par\end{rep}
  %
In addition, codatatypes are guaranteed to contain all values corresponding to
infinite ground constructor terms.
\begin{rep}
In general, this cannot be captured by a
first-order axiomatization, since there may be uncountably many of them.
For example, $\ty{stream}_{\,\ty{int}}$ is isomorphic to the uncountable
function space $\ty{nat} \to \ty{int}$.
\end{rep}

Given a signature $\Sig$, $\thD$ refers to the \emph{theory of
datatypes and codatatypes},
which %in addition to $\Sig$
defines a class of $\Sig$-interpretations $\JJJJ$,
namely the ones that satisfy the %axioms --- existence of all infinite codatatype values cannot be axiomatized
properties mentioned in this section, including (co)induction. The interpretations
in~$\JJJJ$ share the same interpretation for constructor terms and correctly
applied selector terms (up to isomorphism) %renaming of domain elements
but may differ on variables and wrongly applied selector terms.
A formula is $\thD$-\emph{satisfiable} if there
exists an interpretation in $\JJJJ$ that satisfies it.
% otherwise, it is $\thD$-unsatisfiable. (I trust the readers' intelligence when
% it comes to English prefixes like non-, in-, un-, and a-.)
%
For deciding universal formulas, induction can be replaced by the acyclicity
axiom schema, which states that constructor terms cannot be equal to
any of their proper subterms \cite{barrett-et-al-2007}.
Dually, coinduction can be replaced by the uniqueness schema, which
asserts that codatatype values are fully characterized by their %possibly
%infinite
expansion \cite[Theorem 8.1, 2$\,{\Leftrightarrow}\,$5]{rutten-2000}.

%\begin{rep}
% TODO: State more precisely and prove?
%\end{rep}

%    * :
%        enough to consider acyclicity -- no way to specify infinite objects
%        otherwise

%    * codatatypes:
%      * when looking at the universal theory, enough to
%        consider uniqueness -- no way to express that certain infinite
%        acyclic objects do not exist

{\looseness=-1
Some codatatypes are so degenerate as to be finite even though they
have infinite values. A simple example is
\vthinspace$\keyw{codatatype}~\ty{a} = \const{A}(\ty{a})$, whose
unique value is $\MU a.\; \const{A}(a)$. Other specimens are
$\ty{stream}_{\,\ty{unit}\,}$
and
\begin{conf}%
\vthinspace$\keyw{codatatype}~\ty{b} = \const{B}(\ty{b},\: \ty{c},\: \ty{b},\: \ty{unit})
~\keyw{and}~ \ty{c} = \const{C}(\ty{a},\: \ty{b},\: \ty{c})$,
\end{conf}%
\begin{rep}%
\[\begin{aligned}[t]
      \keyw{codatatype}\;\, \ty{b} & \,=\, \const{B}(\ty{b},\: \ty{c},\: \ty{b},\: \ty{unit}) \\[-1\jot]
      \keyw{and}\;\, \ty{c} & \,=\, \const{C}(\ty{a},\: \ty{b},\: \ty{c})
\end{aligned}
\]
\end{rep}%
where \ty{unit} is a datatype with the single constructor $\const{Unity} :
\ty{unit}$. We call such types
\emph{corecursive singletons}. For the decision procedure, it will be
crucial to detect these. %, even if they rarely arise in practice.
A type may also be a corecursive singleton only in some models. If the example
above is altered to make \ty{unit} an uninterpreted type, \ty{b} and \ty{c} will be
singletons precisely when \ty{unit} is interpreted as a singleton.
Fortunately, \begin{rep}given cardinalities for the ordinary types, \end{rep}%
it is easy to characterize this degenerate case.

}
% In other words, all finite corecursive codatatypes are corecursive singletons.

\begin{lemma}%[Corecursive Singletons]%
\label{lem:corecursive-singletons}%
\afterDot
Let $\delta$ be a corecursive codatatype. For any interpretation in $\JJJJ$,
the domain interpreting $\delta$ is either infinite or a singleton. In the
latter case, $\delta$ necessarily has a single constructor, whose arguments have
types that are interpreted as singleton domains.
\end{lemma}

\begin{rep}
\begin{proof}
By definition, the type is equipped with at least one (directly or indirectly)
corecursive constructor $\const{C}$. If it additional has second
corecursive constructor $\const{D}$, it is possible to encode infinitely many
alternation patterns---e.g.,
$\const{C}(\const{D}(\const{C}(\const{C}(\ldots))))$---all of which correspond
to distinct values (by distinctness and injectivity). If the type has a
noncorecursive constructor $\const{E}$, it is possible to create terms of
arbitrary depths---e.g., $\const{C}(\ldots(\const{C}(\const{E}))\ldots)$. In
both cases, there can be no finite models.

Therefore, $\const{C}$ must be the only constructor.
If any of its noncorecursive arguments has a cardinality greater than 1,
it is possible to encode alternation patterns using it---e.g.,
$\const{C}(0,\: \const{C}(1,\: \const{C}(0,\: \const{C}(0,\: \ldots))))$---which
again excludes finite models. Otherwise, the coinduction principle ensures
that the type has at most one value.
\qed
\end{proof}
\end{rep}

%  * assume for simplicity no indirect recursion, but this does not radically
%    change the argument
%  * assume there are at least two values built with C.
%    they must be different at some point, e.g.
%    C(0, C(1, C(0, ...))) and C(0, C(1, C(1, ...)))
%    can use that to create infinitely alternation patterns
%  * leaves us with the case of a single constructor
% * since we are looking at a specific model, we can assume all uninterpreted
%   types are finite datatypes with nullary constructors corresponding to their
%   elements
% * if two
% * either there is only one ``path'' from ctr to itself
%   or at least two;

\section{The Decision Procedure} % for (Co)datatypes}
\label{sec:the-decision-procedure}

Given a fixed signature~$\Sig$,
the decision procedure for the universal theory of (co)data\-types %$\thD$
determines the
$\thD$-satisfiability of finite sets $\Ec$ of literals:\ equalities
and disequalities between $\Sig$-terms,
whose variables are interpreted existentially.
The procedure is formulated as a tableau-like calculus.
%Following the tradition of refutational theorem proving,
Proving a universal quantifier-free conjecture is reduced to showing that
its negation is unsatisfiable. The presentation is
inspired by Barrett et al.\ \cite{barrett-et-al-2007} but
higher-level, using unoriented equations instead of oriented ones.

% whose derivation rules are applied until saturation or contradiction.

To simplify the presentation, we make a few assumptions about
$\Sig$.
First, all codatatypes %in $\Codata$
are corecursive. This is reasonable
because noncorecursive codatatypes can be seen as nonrecursive
datatypes.
Second, all ordinary types have infinite cardinality.
Without quantifiers, the constraints $\Ec$ cannot entail an upper
bound on the cardinality of any uninterpreted type, so it is safe to consider these types
infinite. As for ordinary types interpreted finitely by other theories (e.g.,
bit vectors), each interpreted type having finite cardinality~$n$
can be viewed as a %n enumeration
datatype with $n$~nullary constructors
\cite{barrett-et-al-2007}.

%We will commonly denote tuples of terms $( t_1, \ldots, t_n )$ in bold font, as $\vec t$.
%We assume that all $\Sig$-terms in $\Ec$ are \emph{normalized}, meaning that all subterms of the form $\const{s}^{\,k}_{\negvthinspace j}( \const{C}_{\negvthinspace j}(t_1,\ldots,t_n) )$
%are simplified to $t_k$. All literals added to $\Ec$ are also implicitly
%normalized in this manner.

A derivation rule can be applied to $\Ec$ if %all
the %specified
preconditions are met.
The conclusion either specifies equalities to be added to $\Ec$
or is $\bot$ (contradiction).
% (in which case we call an application of it \emph{nonterminal})
% (in which case we call an application of it \emph{terminal}).
One rule has multiple conclusions, % separated by $\ror$,
denoting branching.
%
An application of a rule is \emph{redundant} if one of its non-$\bot$
conclusions leaves $\Ec$ unchanged.
A \emph{derivation tree} is a %finite
tree whose nodes are finite sets of
equalities, such that child nodes are obtained by a nonredundant application of a
derivation rule to the parent. A derivation tree is \emph{closed} if all of
its leaf nodes are $\bot$. A node is \emph{saturated} if no nonredundant
instance of a rule can be applied to it.

The calculus consists of three sets of rules, given in
\figuresname~\ref{fig:cc-rules} to \ref{fig:split-rule}, corresponding to three
phases. The first phase computes the bidirectional closure of $\Ec$. The second
phase makes inferences based on acyclicity (for datatypes) and uniqueness
(for codatatypes).
The third phase performs case distinctions on constructors for
various terms occurring in $\Ec$.
%Following the conventions from~\cite{},
%The derivation rules are given in \relax{guarded assignment form},
%where
%
The rules belonging to a phase have priority over those of %the
subsequent
phases. The rules are applied until the derivation tree is closed or all leaf nodes
are saturated.

%We assume that all $\Sig$-terms in $\Ec$ are \emph{normalized}, meaning that all subterms of the form $\const{s}^{\,k}_{\negvthinspace j}( \const{C}_{\negvthinspace j}(t_1,\ldots,t_n) )$
%are simplified to $t_i$, and moreover assume that all %additional
%literals added to $\Ec$ are normalized in this manner.
%We present the calculus in three steps.
%In the first step, we compute the bidirectional closure of $\Ec$;
%in the second step, we make inferences based on acyclicity and uniqueness;
%and in the third step, %(when necessary),
%we branch on constructor types for various terms in $\tEc$.
%Following the conventions from~\cite{},
%the derivation rules of our calculus are given in \emph{guarded assignment form},
%where a rule can be applied to $\Ec$ if it meets all of the specified preconditions for $\Ec$.

\paragraph{Phase 1: Computing the Bidirectional Closure \beforefig(\figurename~\ref{fig:cc-rules}).}
In conjunction with \rn{Refl}, \rn{Sym}, and \rn{Trans}, the \rn{Cong} rule computes the congruence (upward) closure,
whereas the \rn{Inject} and \rn{Clash} rules %together
compute the unification (downward) closure.
For unification, %additional
equalities are inferred based on the injectivity of constructors by \rn{Inject},
and failures to unify equated terms are recognized by \rn{Clash}.
The \rn{Conflict} rule recognizes when an equality and its negation both occur in $\Ec$, in which case $\Ec$ has no model.

Let $\tEc$ denote the set of $\Sig$-terms occurring in $\Ec$.
At the end of the first phase, $\Ec$ induces an equivalence
relation over $\tEc$ such that two terms $t$ and $u$ are equivalent if and
only if $t \teq u \in \Ec$.
Thus, we can regard $\Ec$ as a set of
equivalence classes of terms. For a term $t \in \tEc$, we write $\ec{t}$ to
denote the equivalence class of $t$ in $\Ec$.

\paragraph{Phase 2: Applying Acyclicity and Uniqueness \beforefig(\figurename~\ref{fig:ab-rules}).}
%For presentation of the rules in this phase, we rely on a representation of terms in the $\mu$-notation.
%The premises of the rules that make up the second phase refer to
We describe the rules in this phase in terms of a mapping $\Val$
that associates with each equivalence class a $\mu$-term as its representative.
%\rem{Not explicitly constructed, add here?}

Formally, $\mu$-\emph{terms} are defined recursively as being either a variable $x$
or an applied constructor
$\MU x.\: \const{C}( \vec t\, )$ for some $\const{C} \in \Ctr$ and
$\mu$-terms $\vec t$ of the expected types.
The variable~$x$ need not occur free in the $\mu$-binder's body, in which case
the binder can be omitted.
$\FV( t )$ denotes the set of free variables occurring in the $\mu$-term~$t$.
A $\mu$-term is \emph{closed} if %and only if --- by convention, "and only if" is not necessary for *definitions* (but add it back if you disagree)
it contains no free variables. It is \emph{cyclic} if %and only if --- ditto
it contains a bound variable. The $\alpha$\emph{-equiv\-alence} relation $t \aequiv u$
indicates that the $\mu$-terms $t$ and $u$
are syntactically equivalent for some capture-avoiding renaming of $\mu$-bound variables\begin{rep}---e.g.,
%$\MU x.\; \const{C}( x ) \aequiv \MU y.\; \const{C}( y )$ and
$\MU x.\; \const{D}( y, x ) \aequiv \MU z.\; \const{D}( y, z )$,
but
$\MU x.\; \const{C}( x ) \not\aequiv \MU x.\; \const{D}( y, x ) \not\aequiv \MU x.\; \const{D}( z, x )
\not\aequiv \MU y.\; \const{D}( y, x )$\end{rep}.
Two $\mu$-terms can denote the same value despite being $\alpha$-disequivalent---e.g.,
$\MU x.\; \const{S}( x ) \not\aequiv \MU y.\; \const{S}( \const{S}( y ) )$.
%In the \rn{Acyclic} rule,

%Recall that $\mu$-bindings can be used for representing possibly cyclic terms and values.
%(Section~\ref{sec:introduction}).
% for example, $\MU x.\; \const{C}( \const{0}, x )$ represents the cyclic value $\const{C}( \const{0}, \const{C}( \const{0}, \ldots ))$.
%either in the case when $\tau$ is a codatatype or datatype type,
%$\tau \notin \Codata$.

%%% But there are no constants of type $\tau \in \Nondata$ -- at most variables!
%
%For uniformity, we also consider $%\MU x.\;
%\const{c}$ to be a $\mu$-term if $\const{c}$ is a constant of type $\tau \in \Nondata$.
%
%%% I think this should be clear enough from the "Conventions" paragraph at the
%%% end of the introduction.
%%
% which we define recursively:
%if $\const{C} : \vec \tau \rightarrow \upsilon$
%and $\vec v$ are well-typed $\mu$-terms of type $\vec \tau$ under the assumption that $x$ has type $\upsilon$,
%then $\MU x.\; \const{C}( \vec v )$ is a well-typed term of type $\upsilon$.
%
%, or informally are \emph{observationally equivalent}.
%For convenience, we use $\mu$-terms to refer to (classes of) values for both codatatype and datatype terms,
%where in the latter case, a $\mu$-term with a bound variable denotes an infeasible term.
%where the latter case adds the restriction on $\mu$-terms $t$ that no subterms of $t$ contain are bound variables.

The $\mu$-term $\Val[t\typ{\tau}]$ describes a class of $\tau$ values
that $t$ and other members of $t$'s equivalence class can take in models of $\Ec$.
When $\tau$ is a datatype, % \in \Data$,
a cyclic $\mu$-term describes an infeasible class of values.

The mapping $\Val$ is defined as follows.
%
With each equivalence class $\ec{u\typ{\tau}}$, we associate a fresh variable $\Varec{u}\typ{\tau}$
%of the same type as $u$
%not occurring in $\tEc$,
and set $\Val\ec{u} := {\Varec{u}}$, that is to say there are initially no
constraints on the values for any equivalence class $\ec{u}$. The mapping $\Val$
is refined by applying the following unfolding rule exhaustively:\strut
\[
\hbox{\(
\inferrule{
  {\Varec{u}} \in \FV( \Val )
  \betweenantes
  \const{C}( t_1, \ldots, t_n ) \in \ec{u}
  \betweenantes
  \const{C} \in \Ctr
}{
  \Val := \Val [{\Varec{u}} \mapsto \MU {\Varec{u}}.\; \const{C}( {\Varec{t_1}}, \ldots, {\Varec{t_n}} )]
}
\)
%\rn{Unfold}
}
\]
$\FV( \Val )$ denotes the set of free variables occurring in $\Val$'s range,
and $\Val[x \mapsto t]$ denotes the \emph{variable-capturing} substitution of $t$ for
$x$ in $\Val$'s range. It is easy to see that the height of terms produced as a
result of the unfolding is bounded by the number of equivalence classes of
$\Ec$, and thus the construction of $\Val$ will terminate.

\begin{figure}[t!]
\normalsize
\centering
\begin{tabular}{c}
\(
\inferrule{
  t \in \tEc
}{
  \Ec := \Ec,\: t \teq t
}
\)
\rn{Refl}
\qquad
\(
\inferrule{
 t \teq u \in \Ec
}{
 \Ec := \Ec,\: u \teq t
}
\)
\rn{Sym}
\qquad
\(
\inferrule{
  s \teq t,\; t \teq u \in \Ec
}{
  \Ec := \Ec,\: s \teq u
}
\)
\rn{Trans}
\\[5\jot]
\(
\inferrule{
  \vec t \teq \vec u \in \Ec
  \betweenantes
  \const f( \vec t \,),\, \const f( \vec u ) \in \tEc
}{
  \Ec := \Ec,\: \const f( \vec t \,) \teq \const f( \vec u )
}
\)
\rn{Cong}
\qquad
\(
\inferrule{
  t \teq u,\; t \tneq u \in \Ec
}{
  \bot
}
\)
\rn{Conflict}
\\[5\jot]
\(
\inferrule{
  \const{C}( \vec t \,) \teq \const{C}( \vec u ) \in \Ec
}{
  \Ec := \Ec,\: \vec t \teq \vec u
}
\)
\rn{Inject}
\qquad
\(
\inferrule{
  \const{C}( \vec t \,) \teq \const{D}( \vec u ) \in \Ec
  \betweenantes
  \const{C} \not= \const{D}
}{
  \bot
}
\)
\rn{Clash}
\end{tabular}
\vspace*{-2pt} %% TYPESETTING HACK
\caption{\,Derivation rules for bidirectional closure%.
}
\label{fig:cc-rules}
\end{figure}

\begin{figure}[t!]
\vspace*{+6pt} %% TYPESETTING HACK
\normalsize
\centering
\begin{tabular}{@{}c@{}}
\(
\inferrule{
%  t\typ{\delta}
%  \betweenantes
  \delta \in \Data
  \betweenantes
  %\ttpath{\Ec}{\ec{t}}{\ec{t}} \neq \emptyset
  \Val \smash{\ec{t\typ{\delta}}} = \MU x.\; u
  \betweenantes
  x \in \FV( u )
}{
  \bot
}
\)
\rn{Acyclic}
\kern1.5em\( %%% TYPESETTING: should be \quad
\inferrule{
% t, u\typ{\delta}
% \betweenantes
 \delta \in \Codata
 \betweenantes
 %\tpath{\Ec}{\ec{t}} = \tpath{\Ec}{\ec{u}}\sigma \neq \emptyset
 \Val \smash{\ec{t\typ{\delta}}} \aequiv \Val \smash{\ec{u\typ{\delta}}}
}{
 \Ec := \Ec,\: t \teq u
}
\)
\rn{Unique}
\end{tabular}
\vspace*{-2pt} %% TYPESETTING HACK
\caption{\,Derivation rules for acyclicity and uniqueness%.
}
\label{fig:ab-rules}
\end{figure}

\begin{figure}[t!]
\vspace*{+6pt} %% TYPESETTING HACK
\normalsize
\centering
\begin{tabular}{c}
\(
\inferrule{
  t\typ{\delta} \in \tEc
  \betweenantes
  \Ctr^\delta = \{ \const{C}_1, \ldots, \const{C}_m \}
\\
  \bigl( \const s( t ) \in \tEc \text{ and } \const s \in \Sel^\delta \bigr)
  \text{ or }
  \bigl( \delta \in \Data \text{ and } %\left|
  \delta
  %\right|
  \text{ is finite} \bigr)
}{
  \Ec := \Ec,\: \DISC{1} %\ROR
  \ROR \cdots \ROR \Ec := \Ec,\: \DISC{m}
}
\)
\rn{Split}
\\[5\jot]
\(
\inferrule{
  t\typ{\delta}, u\typ{\delta} \in \tEc
  \betweenantes
  \delta \in \Codata
  \betweenantes
  \delta \text{ is a singleton}
}{
  \Ec := \Ec,\: t \teq u
}
\)
\rn{Single}
\end{tabular}
\vspace*{-2pt} %% TYPESETTING HACK
\caption{\,Derivation rules for branching%.
%All pairs of terms whose type has cardinality 1 are entailed to be equal (the degenerate case).
%Constructors must be assigned for all terms $t$ if has a selector is applied to it, or if $t$ has finite datatype type.
}
\label{fig:split-rule}
\end{figure}

\begin{examplex}
Suppose that $\Ec$ contains four distinct equivalence classes $\ec{w}$, $\ec{x}$, $\ec{y}$, and~$\ec{z}$
such that $\const{C}(w,y) \in \ec{x}$ and $\const{C}(z,x) \in \ec{y}$ for some $\const{C} \in \Ctr$.
A possible sequence of unfolding steps is given below, omitting
trivial entries such as $\ec{w} \mapsto {\Varec{w}}$.
%
\begin{enumerate}
%\item \noindent\rlap{Initially:}\phantom{Unfold $\Varec{x}$:\enskip}$\{  \}$
\item \noindent\rlap{Unfold $\Varec{x}$:}\phantom{Unfold $\Varec{x}$:\enskip}$\Val = \{\vthinspace  \ec{x} \mapsto \MU {\Varec{x}}.\; \const{C}( {\Varec{w}},\, {\Varec{y}} ) \vthinspace\}$
\item \noindent\rlap{Unfold $\vthinspace\Varec{y}\negvthinspace$}\phantom{Unfold $\Varec{x}$}:\enskip$\Val = \{\vthinspace  \ec{x} \mapsto \MU {\Varec{x}}.\; \const{C}( {\Varec{w}},\; \MU {\Varec{y}}.\; \const{C}( {\Varec{z}},\, {\Varec{x}} ) ),\;
  \ec{y} \mapsto \MU {\Varec{y}}.\; \const{C}( {\Varec{z}},\, {\Varec{x}} ) \vthinspace\}$
\item \noindent\rlap{Unfold $\Varec{x}$:}\phantom{Unfold $\Varec{x}$:\enskip}$\Val = \{\vthinspace  \ec{x} \mapsto \MU {\Varec{x}}.\; \const{C}( {\Varec{w}},\; \MU {\Varec{y}}.\; \const{C}( {\Varec{z}},\, {\Varec{x}} ) ),\;
  \ec{y} \mapsto \MU {\Varec{y}}.\; \const{C}( {\Varec{z}},\; \MU {\Varec{x}}.\; \const{C}( {\Varec{w}},\, {\Varec{y}} ) ) \vthinspace\}$
\end{enumerate}
%
The resulting $\Val$ indicates that the values for $x$ and $y$ in models of $\Ec$
must be of the forms $\const{C}( {\Varec{w}}, \const{C}( {\Varec{z}},\allowbreak \const{C}( {\Varec{w}},\allowbreak \const{C}( {\Varec{z}},\ldots ))))$
and $\const{C}( {\Varec{z}}, \const{C}( {\Varec{w}}, \const{C}( {\Varec{z}}, \const{C}( {\Varec{w}}, \ldots ))))$,
respectively. %, for some values of ${\Varec{w}}$ and ${\Varec{z}}$.
\xend
\end{examplex}

Given the mapping $\Val$, the \rn{Acyclic} and \rn{Unique} rules work as follows.
For acyclicity, if $\ec{t}$ is a datatype equivalence class
whose values $\Val \ec{t} = \MU x.\; u$ are cyclic
(as expressed by $x \in \FV( u )$),
then $\Ec$ is $\thD$-unsatisfiable.
For uniqueness, if $\ec{t}$, $\ec{u}$ are two codatatype equivalence classes
whose values $\Val \ec{t}$, $\Val \ec{u}$ are $\alpha$-equivalent,
%for a capture-avoiding renaming of $\mu$-bound variables,
then $t \teq u$. Comparison for $\alpha$-equivalence may seem too restrictive, since
$\MU x.\; \const{S}(x)$ and $\MU y.\; \const{S}(\const{S}(y))$ specify the same
value despite being $\alpha$-disequivalent, but the rule will
make progress by discovering that the subterm $\const{S}(y)$ of $\MU y.\;
\const{S}(\const{S}(y))$ must be equal to the entire term\begin{rep}
as demonstrated next\end{rep}.

%% Minor terminology point:
%% A function symbol can have a "type signature" (but we're trying to avoid that terminology);
%% unless it's a constant, it doesn't have a (first-order) type. So I'm rephrasing.

\begin{examplex}
Let $\Ec = \{ x \teq \const{S}(x),\; y \teq \const{S}(\const{S}(y)) \}$.
After phase~1, the equivalence classes %of $\Ec$
are
$\{ x,\, \const{S}(x) \}$,
$\{ y,\, \const{S}(\const{S}(y)) \}$,
and $\{ \const{S}(y) \}$.
Constructing $\Val$ yields
\begin{align*}
\Val \ec{x} & \,=\, \MU {\Varec{x}}.\; \const{S}(\Varec{x}) &
\kern0.25em %% TYPESETTING
\Val \ec{y} & \,=\,
\MU {\Varec{y}}.\; \const{S}(\smash{\MU {\VAREC{\const{S}(y)}}}.\; \const{S}( {\Varec{y}} ) )
\kern0.25em %% TYPESETTING
 &
\Val \ec{\const{S}(y)} & =
\MU {\VAREC{\const{S}(y)}}.\; \const{S}(\MU {\Varec{y}}.\; \const{S}( {\VAREC{\const{S}(y)}} ) )
\end{align*}
Since $\Val \ec{y} \aequiv \Val \ec{\const{S}(y)}$,
%for the renaming
%$\{ {\Varec{y}} \mapsto {\VAREC{\const{S}(y)}}, {\VAREC{\const{S}(y)}} \mapsto {\Varec{y}} \}$,
the \rn{Unique} rule applies to derive $y \teq \const{S}(y)$.
%Intuitively, this equality holds, since the values of $y$ and $\const{S}(y)$
%are equivalent.
%
At this point, phase~1 is activated again, % to XXX the equivalence classes,
yielding
$\{ x,\, \const{S}(x) \}$ and
$\{ y,\, \const{S}(y),\allowbreak\, \const{S}(\const{S}(y)) \}$.
The mapping $\Val$ is updated accordingly:
\begin{conf}%
$\Val \ec{y} \,=\, \MU {\Varec{y}}.\; \const{S}({\Varec{y}})$.
\end{conf}%
\begin{rep}
\begin{align*}
\Val \ec{x} & \,=\, \MU {\Varec{x}}.\; \const{S}(\Varec{x})
&
\Val \ec{y} & \,=\, \MU {\Varec{y}}.\; \const{S}({\Varec{y}})
\end{align*}
\end{rep}
Since $\Val \ec{x} \aequiv \Val \ec{y}$,
%for the renaming
%$\{ {\Varec{y}} \mapsto {\VAREC{\const{S}(y)}}, {\VAREC{\const{S}(y)}} \mapsto {\Varec{y}} \}$,
\rn{Unique} can finally derive $x \teq y$.
\xend
\end{examplex}

\paragraph{Phase 3: Branching \beforefig(\figurename~\ref{fig:split-rule}).}
If a selector is applied to a term $t$, or if $t$'s type is a finite datatype,
$t$'s equivalence class must contain a
$\delta$ constructor term.
This is enforced in the third phase by the \rn{Split} rule.
Another rule, \rn{Single}, focuses on the degenerate case where two
terms %from $\tEc$
have the same corecursive singleton type
%(Section~\ref{sec:co-datatypes})
and are therefore equal. Both
\rn{Split}'s finiteness assumption %on the datatype
and \rn{Single}'s singleton
constraint %on the codatatype
can be evaluated statically
based on a recursive computation of the cardinalities of the
constructors' argument types.
%This step relies on the assumption that uninterpreted types are infinite.

\paragraph{Correctness.}
Correctness means that if there exists a closed derivation tree with root node
$\Ec$, then $\Ec$ is $\thD$-unsatisfiable; and if there exists
a derivation tree with root node $\Ec$ that contains a saturated node, then
$\Ec$ is $\thD$-satisfiable.

\begin{theorem}[Termination]%
\label{thm:t}%
\afterDot
All derivation trees are finite.
\end{theorem}
\begin{proof}
Consider a derivation tree with root node $\Ec$. Let $D \subseteq \tEc$ be the
set of terms whose types are finite datatypes, and let $\SSS \subseteq \tEc$ be
the set of terms occurring as arguments to selectors. For each term $t \in
D$\negvthinspace, let
\begin{conf}
$\smash{\SSSS 0_t} = \{ t \}$ and
$\smash{\SSSS {i+1}_t} = \smash{\SSSS i_t \mathrel\cup \{ \const s( u ) \mid u\typ{\delta} \in \SSSS i_t{,}}~\allowbreak\smash{ \delta \in \Data{,}~ \left|\delta\right| \text{ is finite}{,}~ \const{s} \in \Sel^{\delta}}  \}$,
\end{conf}%
\begin{rep}
\begin{align*}
\smash{\SSSS 0_t} & = \{ t \}
& \smash{\SSSS {i+1}_t} & = \smash{\SSSS i_t \mathrel\cup \{ \const s( u ) \mid u\typ{\delta} \in \SSSS i_t{,}~ \delta \in \Data{,}~ \left|\delta\right| \text{ is finite}{,}~ \const{s} \in \Sel^{\delta}}  \}
\end{align*}
\end{rep}%
%for $i \geq 0$, let ;
and let $\SSSS \AST_t$ be the limit of this sequence.
This is a finite set for each $t$, 
because all chains of selectors applied to $t$ are finite.
%because in the absence of (co)recursion
%one eventually reaches a point where no selector can be applied.
%%% TODO: Clarify above step in report.
Let $\SSSS \AST$ be the union of all sets $\SSSS \AST_t$ where $t \in D$,
and let $\tcEc$ be the set of subterms of
%
$\Ec \mathrel\cup \{ \smash{\const C_{\negvthinspace j}\bigl( \const s^1_{\negvthinspace j}( t ), \ldots, \const s^{\smash{\,n_{\negvthinspace j}}}_{\negvthinspace j}( t ) \bigr)} \mid \smash{t\typ{\delta} \in \SSS \mathrel\cup \SSSS \AST{,}~ \const C_{\negvthinspace j} \in \smash{\Ctr^\delta}} \}$.
%
In a derivation tree with root node~$\Ec$,
it can be shown by induction
on the rules of the calculus that each non-root node $\Fc$ is such that
$\Terms(\Fc) \subseteq \tcEc$, and hence contains an equality between two terms from $\tcEc$ not occurring in its parent node.
Thus, the depth of a branch in a derivation tree with root node $\Ec$ is at most $\left| \smash{\tcEc}\vphantom{X} \right|{\!}^2$,
which is finite since $\tcEc$ is finite.
\qed
\end{proof}

\begin{theorem}[Refutation Soundness]%
\label{thm:rs}%
\afterDot
If there exists a closed derivation tree with root node\/ $\Ec$, then\/ $\Ec$ is $\thD$-unsatisfiable.
\end{theorem}
\begin{proof}
The proof is by structural induction on the derivation tree with root node $\Ec$.
If the tree is an application of \rn{Conflict}, \rn{Clash}, or \rn{Acyclic},
then $\Ec$ is $\thD$-unsatisfiable.
For \rn{Conflict}, this is a consequence of equality reasoning.
For \rn{Clash}, this is a consequence of distinctness.
For \rn{Acyclic}, the construction of $\Val$ indicates that the class of values that term~$t$ can take in models of $\Ec$ is infeasible.
If the child nodes of $\Ec$ are closed derivation trees
whose root nodes are the result of applying \rn{Split} on $t\typ{\delta}$,
by the induction hypothesis $\Ec \mathrel\cup t \teq \smash{\const C_{\negvthinspace j}\bigl( \const s^1_{\negvthinspace j}( t ), \ldots, \const s^{\,\smash{n_{\negvthinspace j}}}_{\negvthinspace j}( t ) \bigr)}$ is
$\thD$-unsatisfiable
for each $\const C_{\negvthinspace j} \in \smash{\Ctr^\delta}$.
Since by exhaustiveness, all models of $\thD$ entail exactly one
$\DISC{\negvthinspace j}$,
$\Ec$ is $\thD$-unsatisfiable.
Otherwise, the child node of $\Ec$ is a closed derivation tree
whose root
node $\Ec \mathrel\cup t \teq u$ is obtained by applying one of the rules \rn{Refl}, \rn{Sym}, \rn{Trans}, \rn{Cong}, \rn{Inject}, \rn{Unique}, or \rn{Single}.
In all these cases, $\Ec \models_{\thD} t \teq u$.
For \rn{Refl}, \rn{Sym}, \rn{Trans}, \rn{Cong}, this is a consequence of equality reasoning.
For \rn{Inject}, this is a consequence of injectivity.
For \rn{Unique}, the construction of $\Val$ indicates that the values of $t$ and $u$ are equivalent in all models of $\Ec$.
For \rn{Single}, $t$~and~$u$ must have the same value since the cardinality of their type is one.
By the induction hypothesis, $\Ec \mathrel\cup t \teq u$ is $\thD$-unsatisfiable
and thus $\Ec$ is $\thD$-unsatisfiable.
\qed
\end{proof}

It remains to show the converse of the previous theorem: If a derivation tree
with root node $\Ec$ contains a saturated node, then $\Ec$ is
$\thD$-satisfiable.
The proof relies on a specific interpretation $\J$ that satisfies $\Ec$.

First, we define the set of interpretations of the theory $\thD$,
which requires custom terminology concerning $\mu$-terms.
Given a $\mu$-term $t$ with subterm $u$,
the \emph{expansion of $u$ with respect to $t$} is the $\mu$-term $\expand{t}{u}^\emptyset$,
abbreviated to
$\expand{t}{u}$, as returned by the function
%where $x \leadsto \MU x.\; \const{C}( \vec u ) \in t$ indicates that $\MU x.\;
%\const{C}( \vec u )$ is the subterm of $t$ that binds this occurrence of
%variable $x$:
\[\begin{array}{@{}r@{}c@{}l@{}}
\expand{t}{x}^B & {} \,=\, {} &
\textstyle\relax{\begin{cases}
    x & \text{if $x \in B$}   \\[-\jot]
    \MU x.\; \const{C}\smash{\bigl( \expand{t}{ \vec u}^{\smash{B \,\uplus\, \{ x \}}}\bigr)} & \text{if
    $\MU x.\; \const{C}( \vec u )$ binds this occurrence of $x \notin B$ in $t$}
\end{cases}} ~ \\
\expand{t}{\MU x.\; \const{C}( \vec u )}^B & = &
\textstyle\begin{cases}
    x & \text{if $x \in B$} \\[-\jot]
    \MU x.\; \const{C}\smash{\bigl( \expand{t}{ \vec u }^{\smash{B \,\uplus\, \{ x \}}}\bigr)} & \text{otherwise}
\end{cases}
\end{array}\]
The recursion will eventually terminate because each recursion adds one bound
variable to $B$ and there are finitely many distinct bound variables in a $\mu$-term.
Intuitively, the expansion of a subterm is a stand-alone
$\mu$-term that denotes the same value as the original subterm---e.g.,
$\expand{\MU x.\: \const{C}(\MU y.\: \const{D}(x))}{\MU y.\: \const{D}(x)} =
 \MU y.\: \const{D}(\MU x.\: \const{C}(y))$.
%
%\rem{TODO: add small examples}
%\rem{TODO: explain why terminates}

The $\mu$-term $u$
is a \emph{self-similar subterm} of~$t$ if
$u$ is a proper subterm of $t$,
$t$ and $u$ are of the forms $\MU x.\; \const{C}( t_1, \ldots, t_n )$
and $\MU y.\; \const{C}( u_1, \ldots, u_n )$,
and $\expand{t}{ t_k } \aequiv \expand{t}{ u_k }$ for all $k$. % \in [n]
The $\mu$-term $t$ is \emph{normal} if it does not contain %any
self-similar subterms
and all of its proper subterms are also normal.
Thus, $t = \MU x.\; \const{C}( \MU y.\; \const{C}( y ) )$ is not normal
because $\MU y.\; \const{C}( y )$ is a self-similar subterm of $t$.
Their arguments have the same expansion with respect to~$t$:
$\expand{t}{\MU y.\; \const{C}( y )} =
\MU y.\; \const{C}\bigl( \expand{t}{y}^{\smash{\{y\}}} \bigr) =
\MU y.\; \const{C}( y )$
is $\alpha$-equivalent to
$\expand{t}{y} =
\MU y.\; \const{C}\bigl( \expand{t}{y}^{\smash{\{y\}}} \bigr) =
\MU y.\; \const{C}( y )$.
The term $u = \MU x.\; \const{C}( \MU y.\; \const{C}( x ) )$ is also not normal,
since $\MU y.\; \const{C}( x )$ is a self-similar subterm of $u$,
noting that
%
$\expand{u}{\MU y.\; \const{C}( x )}
 = \MU y.\; \const{C}\bigl( \expand{u}{x}^{\smash{\{y\}}} \bigr)
 = \MU y.\; \const{C}\bigl( \expand{u}{\MU x.\allowbreak\; \const{C}( \MU y.\allowbreak\; \const{C}( x ) ) }^{\smash{\{y\}}} \bigr)
 = \MU y.\; \const{C}\bigl( \MU x.\allowbreak\; \const{C}( \expand{u}{\MU y.\allowbreak\; \const{C}( x )}^{\smash{\{x,y\}}} ) \bigr)
 = \MU y.\allowbreak\; \const{C}( \MU x.\allowbreak\; \const{C}( y ) )$
is $\alpha$-equivalent to $\expand{u}{x} = u$.
% for the renaming $\{ x \mapsto y, y \mapsto x \}$.

For any $\mu$-term $t$ of the form $\MU x.\; \const C( \vec u )$,
its \emph{normal form} $\nf{t}$ is obtained
by replacing all of the self-similar subterms of $t$ with $x$
%and by replacing all of the abnormal subterms of $t$ with their corresponding normal forms.
and by recursively normalizing the other subterms.
For variables, $\nf{x} = x$.
Thus, $\nf{\MU x.\; \const{C}( \MU y.\; \const{C}( x ) )} = \MU x.\; \const{C}( x )$.

\begin{rep}
\begin{lemma}
\label{lem:mu-norm-arg}
If the $\mu$-terms $\vec u$ are in normal form and $t = \nf{\MU x.\; \const{C}( \vec u )} = \MU x.\; \const{C}( \vec w )$,
then $\vec u \vsim \expand{t}{\vec w}$\vthinspace.
\end{lemma}
% TODO:
%\begin{proof}
%\qed
%\end{proof}

\begin{lemma}
\label{lem:mu-norm-interp}
If the $\mu$-term $t$ is normal and $u$ is a subterm of $t$, then $\expand{t}{u}$ is normal.
\end{lemma}
% TODO:
%\begin{proof}
%\qed
%\end{proof}

%\begin{lemma}
%\label{lem:mu-cong}
%If $t_u = \MU x.\; \const{C}( \vec u )$ and $t_w = \MU y.\; \const{C}( \vec w )$ are in normal form,
%then $\expand{t_u}{ \vec u } \vsim \expand{t_w}{ \vec w }$ if and only if $t_u \vsim t_w$.
%\end{lemma}
%\begin{proof}
%\rem{TODO}
%\qed
%\end{proof}
\end{rep}

We now define the class of interpretations for $\thD$.
$\interp{\J}{\tau}$ denotes the interpretation type $\tau$ in $\J$---that is, a
nonempty set of domain elements for that type.
$\interp{\J}{\const{f}}$ denotes the interpretation of a function $\const{f}$ in $\J$.
If $\const{f} : \tau_1 \times \cdots \times \tau_n \to \tau$,
then $\interp{\J}{\const{f}}$ is a total function from $\interp{\J}{\tau_1} \times \cdots \times \interp{\J}{\tau_n}$ to $\interp{\J}{\tau}$.
%To simplify the presentation,
All types are interpreted as sets of $\mu$-terms, but
only values of types in $\Codata$ may contain cycles.
%Since all values of types in $\Data$ and $\Nondata$ are acyclic,
%a typical representation for these values can be obtained by
%dropping each of its $\mu$-bindings.
%For example, the $\mu$-term $\MU x.\; \const{Z}$ representing the integer constant zero
%is simply $\const{Z}$ after dropping the $\mu$-binding of $x$.

\begin{definition}[Normal Interpretation]
\afterDot%
\label{def:norm-interpretation}%
\rm
An interpretation $\J$ is \emph{normal} if these conditions are met:
\begin{enumerate}
\item
For each type $\tau$,
%%% sets *contain* elements, they *include* other sets
$\interp{\J}{\tau}$ includes a maximal set of closed normal $\mu$-terms of that type that are
unique up to $\alpha$-equivalence and acyclic if $\tau \notin \Codata$.\strut

\item
For each constructor term $\vphantom{s^k}\const{C}( \vec t \,)$ of type $\tau$,
$\interp{\J}{\const{C}}( \interp{\J}{\vec t} )$ is the value
in $\interp{\J}{\tau}$ that is $\alpha$-equivalent to
$\nf{\MU x.\; \const{C}( \interp{\J}{\vec t} )}$, where $x$ is fresh.\strut

\item
For each selector term $\const s^k_{\negvthinspace j}( t )$ of type $\tau$, % for some $\vec u$,
if $\interp{\J}{t}$ is $\MU x.\; \const{C}_{\negvthinspace j}( \vec u )$,
then $\J\negvvthinspace( \const s^k_{\negvthinspace j} )\bigl( \interp{\J}{t} \bigr)$ is the value
in $\interp{\J}{\tau}$ that is $\alpha$-equivalent to
$\expand{\interp{\J}{t}}{u_k}$.
\end{enumerate}
\end{definition}

Not all normal interpretations are models of codatatypes, because
models must contain all possible infinite terms, not only cyclic ones. However,
acyclic infinite values are uninteresting to us, and for quantifier-free
formulas it is trivial to extend any normal interpretation with extra
domain elements to obtain a genuine model if desired. %Hence, we focus on normal interpretations.

\pagebreak[2]

When constructing a model $\J$ of $\Ec$,
it remains only to specify how $\J$ interprets wrongly applied selector terms and variables.
For the latter, this will be based on the mapping $\Val$ computed in phase~2 of the calculus.

{\looseness=-1
First, we need the following definitions.
We write $t \vsimv{x} u$ if $\mu$-terms $t$ and $u$ are syntactically equivalent
for some renaming that avoids capturing any variable other than~$x$.
For example,
$\MU x.\; \const{D}( x ) \vsimv{y} \MU x.\; \const{D}( y )$
(by renaming $y$ to $x$),
$\MU x.\; \const{C}( x, x ) \vsimv{x} \MU y.\; \const{C}( x, y )$, and
$\MU x.\; \const{C}( z, x ) \vsimv{z} \MU y.\; \const{C}( z, y )$,
but
$\MU x.\; \const{D}( x ) \not\vsimv{x} \MU x.\; \const{D}( y )$ and
$\MU x.\; \const{C}( x, x ) \not\vsimv{y} \MU y.\; \const{C}( x, y )$.
For a variable $x\typ{\tau}$ and a normal interpretation $\J$,
we let $\BAD_{\smash{\J}}^x( \Val )$ denote the set consisting of all values
$v \in \interp{\J}{\tau}$
such that $v \vsimv{x} \expand{t}{u}$ for some subterm $u$ of a term $t$
occurring in the range of $\Val$. This set describes shapes of terms to avoid
when assigning a $\mu$-term to $x$.

}

The \emph{completion} $\ValC$ of $\Val$ for a normal interpretation $\J$
assigns values from $\J$ to unassigned variables in the domain of $\Val$.
We construct $\ValC$ by initially setting $\ValC := \nf{\Val}$
and by exhaustively applying the following rule:%
\[
\hbox{\(
\inferrule{
  {\Varec{x}}\typ{\tau} \in \FV( \ValC )
  \betweenantes
  \MU {\Varec{x}}.\; t \aequiv v
  \betweenantes
  v \in \interp{\J}{\tau}
  \betweenantes
  v \notin \BAD_{\smash{\J}}^{\Varec{x}}(\ValC)
}{
  \ValC := \nf{\ValC [ {\Varec{x}} \mapsto \MU {\Varec{x}}.\; t ]}
}
\)
%\rn{Assign}
}
\]
%
Given an unassigned variable in $\ValC$, this rule assigns it a fresh
value---one that does not occur in $\BAD_{\smash{\J}}^{\Varec{x}}(\ValC)$
modulo $\alpha$-equivalence---excluding
not only existing terms in the range of $\ValC$ but also
terms that could emerge as a result of the update.
%\rem{TODO: What guarantees that such a value exists? (Answer: infinitely many values
%of codatatypes, of which only finitely many are in
%$\BAD_{\smash{\J}}^{\Varec{x}}(\ValC)$?)}
Since this update removes one
variable from $\FV( \ValC )$ and does not add any variables to $\FV( \ValC
)$, the process eventually terminates. We normalize all terms in
the range of $\ValC$ at each step.

%The last precondition is admittedly rather technical.
To ensure disequality literals are satisfied by an interpretation based on $\ValC$,
it suffices that $\ValC$ is injective modulo $\alpha$-equivalence.
This invariant holds initially, and
the last precondition in the above rule ensures that it is maintained.
%More precisely,
The set $\BAD_{\smash{\J}}^{\Varec{x}}(\ValC)$ is an overapproximation of the values
that, when assigned to ${\Varec{x}}$, will cause values in the range of $\ValC$ to become $\alpha$-equivalent.
%This precondition is not strictly necessary in practice, 
%as one could compute $\nf{\ValC [ {\Varec{x}} \mapsto \MU {\Varec{x}}.\; t ]}$ optimistically and check
%injectivity afterward, repeating until a suitable value is found.
For infinite codatatypes, it is always possible to find fresh values $v$
because $\BAD_{\smash{\J}}^{\Varec{x}}(\ValC)$ is a finite set.
%Injectivity is important to satisfy disequality literals.
%For infinite codatatypes, it is always possible to find suitable values $v$.
%For example,
%if the longest cycle in $\BAD_{\smash{\J}}^{\Varec{x}}(\ValC)$ is of length $n$, a fresh value
%can be constructed with a cycle of length $n + 1$.

\begin{examplex}
Let $\delta$ be a codatatype with the constructors $\const{C},\, \const{D},\, \const{E} :\delta \rightarrow \delta$.
Let $\Ec$ be the set
$\{
u \teq \const{C}( z ),\;
v \teq \const{D}( z ),\;
w \teq \const{E}( y ),\;
x \teq \const{C}( v ),\;
z \not\teq v
\}$.
After applying the calculus to saturation on $\Ec$,
the mapping $\Val$ is as follows:
%\[\begin{array}{r@{}c@{}l@{\;}l@{\qquad}r@{}c@{}r@{\;}l@{}l@{\;}l}
%\[\begin{array}{r@{}c@{}l@{\;}l@{\qquad}r@{}c@{}r@{\;}l@{}l@{\;}l@{\qquad}r@{}c@{}r@{\;}l}
\[\begin{array}{r@{}c@{}r@{}l@{\qquad}r@{}c@{}r@{}l@{\qquad}r@{}c@{}l}
\Val \ec{u} & {}\,=\,{} & \MU {\Varec{u}}.\; & \const{C}( {\Varec{z}} ) &
\Val \ec{w} & = & \MU {\Varec{w}}.\; & \const{E}( {\Varec{y}} ) &
\Val \ec{y} & {}\,=\,{} & {\Varec{y}} \\
\Val \ec{v} & = & \MU {\Varec{v}}.\; & \const{D}( {\Varec{z}} ) &
\Val \ec{x} & {}\,=\,{} & \MU {\Varec{x}}.\; & \const{C}( \MU {\Varec{v}}.\; \const{D}( {\Varec{z}} ) ) &
\Val \ec{z} & = & {\Varec{z}}
\end{array}\]
%
To construct a completion $\ValC$, we must choose values for ${\Varec{y}}$ and ${\Varec{z}}$,
which are free in~$\Val$.
Modulo $\alpha$-equivalence, $\smash{\BAD_{\smash{\J}}^{{\Varec{z}}}( \Val )} =
%\begin{array}{c}
\{
\MU a.\: \const{C}( a ),\:
\MU a.\: \const{D}( a ),\allowbreak\:
\MU a.\: \const{C}( \const{D}( a )),\allowbreak\:
\const{C}( \MU a.\: \const{D}( a ))
\}
%\end{array}
$.
Now consider a normal interpretation $\J$ that evaluates variables in $\Ec$ based on~$\Val$:
$\interp{\J}{u} = \Val \ec{u}$, $\interp{\J}{v} = \Val \ec{v}$, and so on.
Assigning a value for $\Val \ec{z}$ that is $\alpha$-equivalent to a value in $\BAD_{\smash{\J}}^{{\Varec{z}}}( \Val )$
may cause values in the range of $\Val$ to become $\alpha$-equivalent,
which in turn may cause $\Ec$ to be falsified by $\J$.
For example, assign $\MU {\Varec{z}}.\; \const{D}( {\Varec{z}} )$ for ${\Varec{z}}$.
After the substitution, $\Val \ec{v} = \MU {\Varec{v}}.\; \const{D}( \MU {\Varec{z}}.\; \const{D}( {\Varec{z}} ) )$,
which has normal form $\MU {\Varec{v}}.\; \const{D}( {\Varec{v}} )$,
which is $\alpha$-equivalent to $\MU {\Varec{z}}.\; \const{D}( {\Varec{z}} )$.
However, this contradicts the disequality $z \not\teq v$ in~$\Ec$.
On the other hand, if the value assigned to ${\Varec{z}}$ is fresh, %not $\alpha$-equivalent to any term in $\BAD_{\smash{\J}}^{{\Varec{z}}}( \Val )$,
the values in the range of $\Val$ remain $\alpha$-disequivalent.
We can assign a value such as
$\MU {\Varec{z}}.\; \const{E}( {\Varec{z}} )$,
$\MU {\Varec{z}}.\; \const{D}( \const{C}( {\Varec{z}} ) )$, or
$\MU {\Varec{z}}.\; \const{C}( \const{C}( \const{D}( {\Varec{z}} ) ) )$
to ${\Varec{z}}$.
\begin{rep}Legal substitutions for ${\Varec{z}}$ may cause the range of $\Val$ to contain abnormal terms.
For example, after assigning $\MU {\Varec{z}}.\; \const{D}( \const{C}( {\Varec{z}} ) )$ to $\Varec{z}$, we have
$\Val \ec{u} = \MU {\Varec{u}}.\; \const{C}( \MU {\Varec{z}}.\; \const{D}( \const{C}( {\Varec{z}} ) ) )$,
with normal form
$\MU {\Varec{u}}.\; \const{C}( \MU {\Varec{z}}.\; \const{D}( {\Varec{u}} ) )$.\end{rep}
\xend
\end{examplex}

In the following lemma about $\ValC$,
$\vphantom{\Bigl(_j}\muvar( t ) = \relax{\begin{cases}
  t & \text{if $t$ is a variable} \\[-\jot]
  x & \text{if $t$ is of the form $\MU {x}.\; u$.}
\end{cases}}$

\pagebreak[2]

%%% I think it's enough to point this out clearly in the introduction.
%%% At least, that always worked fine for my other papers. :)
%
%\begin{conf}
%The proof of this lemma can be found in the extended version of this paper.
%\end{conf}

\begin{lemma}
\label{lem:interpretation-completion}
If $\Val$ is constructed for a saturated set $\Ec$
and $\ValC$ is a completion of $\Val$ for a normal interpretation $\J$, the following properties hold:
\begin{enumerate}
\item[\rm (1)] $\ValC \ec{x\typ{\tau}}$ is $\alpha$-equivalent to a value in $\interp{\J}{\tau}$.
\item[\rm (2)]
$\ValC \ec{x} = \expand{\ValC \ec{y}}{ t }$ for
all subterms $t$ of $\ValC \ec{y}$ with $\muvar(t) = {\Varec{x}}$.
\item[\rm (3)]
$\ValC \ec{x} \vsim \ValC \ec{y}$ if and only if $\ec{x} = \ec{y}$.
\end{enumerate}
\end{lemma}
\begin{rep}
\begin{proof}
To show (1), we first show that $\ValC$ contains no free variables.
Assume by contradiction that $\ValC$ contained a free variable ${\Varec{y}}$ for some $\ec{y}$ of type $\tau$.
Then it must be the case that $\ec{y}$ does not contain a constructor term,
or else ${\Varec{y}}$ would not occur as a free variable in $\Val$.
Consider the case when $\tau$ is finite.
By assumption, $\tau \notin \Nondata$.
Since \rn{Split} does not apply to $\Ec$, we have $\tau \notin \Data$.
If $\tau \in \Codata$, then $\tau$ is corecursive by assumption, and by Lemma~\ref{lem:corecursive-singletons},
the cardinality of $\tau$ must be one.
Since \rn{Single} does not apply, % to $\Ec$,
there is only one equivalence class of type $\tau$ in $\Ec$,
and thus there are no terms in $\BAD_{\smash{\J}}^{\Varec{y}}(\ValC)$ of type $\tau$.
This is a contradiction, since %our %model
completion can assign the value in the domain of $\tau$ to ${\Varec{y}}$.
Now, consider the case when $\tau$ is infinite.
This is also a contradiction,
since there are only a finite number of closed terms in $\BAD_{\smash{\J}}^{\Varec{y}}(\ValC)$,
and thus %our %model
completion can assign a value not occurring in $\BAD_{\smash{\J}}^{\Varec{y}}(\ValC)$ to ${\Varec{y}}$.
By construction, $\ValC \ec{x}$ is normal.
Since \rn{Acyclic} does not apply, $\Val \ec{x}$ is acyclic when $\tau \in \Data$.
Moreover, the construction of $\ValC$ applies substitutions of the form
$\{ {\Varec{y}} \mapsto t \}$, where $t$ is acyclic when the type of ${\Varec{y}}$ is not a codatatype.
Thus, $\ValC \ec{x}$ is acyclic when $\tau \notin \Codata$.
Therefore, by definition, $\ValC \ec{x}$ is $\alpha$-equivalent to a value in $\interp{\J}{\tau}$.

We first show that (2) and (3) hold initially for $\Val$.
For all equivalence classes $\ec{z}$,
each pair of constructor terms $\const{C}_{\negvthinspace j}( \vec t \,)$ and $\const{C}_{\negvthinspace\jPrime}( \vec u )$ in $\ec{z}$
are such that $j = \jPrime$ since \rn{Clash} does not apply, % to $\Ec$,
and are such that $\ec{ \vec t } = \ec{ \vec u }$ since \rn{Inject} does not apply. % to $\Ec$.
Thus,
$\Val$ was constructed by applying a sequence of substitutions
where all substitutions for variables ${\Varec{z}}$
were uniquely of the form $\{ {\Varec{z}} \mapsto \const{C}_{\negvthinspace j}( {\Varec{t_1}}, \ldots, {\Varec{t_n}} ) \}$
when $\ec{z}$ contains a constructor $\const{C}_{\negvthinspace j}( t_1, \ldots, t_n )$.
Say $\Val \ec{y}$ has a subterm $t$ where $\muvar(t) = {\Varec{x}}$.
Both $\Val \ec{x}$ and the subterm $t$ of $\Val \ec{y}$ were
constructed by applying a sequence of substitutions of the form mentioned above to ${\Varec{x}}$.
Moreover, free variables ${\Varec{z}}$ in $t$ that are bound in $\Val \ec{y}$ are interpreted in
the expansion of $\expand{\Val \ec{y}}{ t }$ as a term
constructed by a sequence of substitutions of the form mentioned above to ${\Varec{z}}$. %\rem{TODO: more detail}
Thus, we have
$\expand{\Val \ec{y}}{ t } = \expand{\Val \ec{x}}{ \Val \ec{x} } = \Val \ec{x}$, and thus (2) holds for $\Val$.
Property (3) holds for $\Val$ since \rn{Unique} does not apply.

We now show that $\Val = \nf{\Val}$.
Assume by contradiction $\Val \ec{x} \neq \nf{\Val \ec{x}}$ for some $\Val \ec{x}$ of minimal size.
We have that $\Val \ec{x}$ is of the form $\MU {\Varec{x}}.\; \const{C}( t_1, \ldots, t_n )$.
Due to the construction of $\Val$,
we know $\ec{x}$ contains a constructor $\const{C}( z_1, \ldots, z_n )$ and $\muvar( t_i ) = z_i$ for some $i$ % \in [n]$.
Since $\Val \ec{x}$ is a minimal, it
contains a subterm of the form $\MU {\Varec{y}}.\; \const{C}( u_1, \ldots, u_n )$
where $\expand{\Val \ec{x}}{ t_i } \aequiv \expand{\Val \ec{x} }{ u_i }$ for some $i$. % \in [n]$.
Due to the construction of $\Val$, $\ec{y}$ contains a constructor $\const{C}( w_1, \ldots, w_n )$ and $\muvar( u_i ) = w_i$ for some $i$. % \in [n]$.
Since \rn{Cong} does not apply, % to $\Ec$
we have $\ec{w_{\negvthinspace j}}$ and $\ec{z_{\negvthinspace j}}$ are distinct for some $j$.
By (2), $\expand{\Val \ec{x}}{ t_{\negvthinspace j} } = \Val \ec{z_{\negvthinspace j}}$
and $\expand{\Val \ec{x}}{ u_{\negvthinspace j} } = \Val \ec{w_{\negvthinspace j}}$,
which are not $\alpha$-equivalent by (3), contradicting the fact that $\MU {\Varec{y}}.\; \const{C}( u_1, \ldots, u_n )$ is a self-similar subterm of $\Val \ec{x}$.
Thus, $\Val = \nf{\Val}$, and (2) and (3) hold for $\nf{\Val}$.

We now show that if (2) and (3) hold for some $\Val_1$,
they also hold for $\nf{\Val_1 \sigma}$,
where $\sigma$ is a substitution of the form $\{ {\Varec{x}} \mapsto \MU{\Varec{x}}.\; t \}$,
${\Varec{x}} \in FV( \Val_1 )$,
and $\MU {\Varec{x}}.\; t$ is not $\alpha$-equivalent to a term in $\BAD_{\smash{\J}}^{\Varec{x}}(\Val_1)$.
To show (2),
by assumption of (2) on $\Val_1$, we have $\expand{\Val_1 \ec{y}}{ u } = \Val_1 \ec{x}$ for
all subterms $u$ of $\Val_1 \ec{y}$ where $\muvar(t) = {\Varec{x}}$.
Thus, $\expand{\Val_1 \sigma \ec{y}}{ u } = \Val_1 \sigma \ec{x}$ and
$\expand{\nf{\Val_1 \sigma} \ec{y}}{ u } = \nf{\Val_1 \sigma} \ec{x}$.
To show (3),
consider two distinct equivalence classes $\ec{y}$ and $\ec{z}$,
and assume by contradiction that $\nf{\Val_1 \sigma} \ec{y} \aequiv \nf{\Val_1 \sigma} \ec{z}$.
Due to (3) for $\Val_1$,
$\ec{y}$ and $\ec{z}$ must have (minimal) subterms
where $t_1$ occurs in $\Val_1 \ec{y}$ the same position (call it $p$) as $t_2$ occurs in $\Val_1 \ec{z}$,
and $t_1 \not\aequiv t_2$.
%We first show that $\nf{\Val_1 \sigma} \ec{y}$ (and for the same reasons, $\nf{\Val_1 \sigma} \ec{z}$)
%also contains position $p$.
%If this were not the case,
%then without loss of generality we may assume
%$\Val_1 \sigma \ec{y}$ contains a self-similar subterm
%$\MU {\Varec{w}}.\; u$ that strictly contains $t_1 \sigma$.
If $t_1$ (resp. $t_2$) is a free variable that is not ${\Varec{x}}$,
then $\nf{\Val_1 \sigma} \ec{y}$ (resp. $\nf{\Val_1 \sigma} \ec{z}$)
contains $t_1$ (resp. $t_2$) at position $p$,
and $\nf{\Val_1 \sigma} \ec{z}$ (resp. $\nf{\Val_1 \sigma} \ec{y}$) does not.
If $t_1$ is of the form $\MU w_1.\; \const{C}( \vec t \,)$
and $t_2$ is of the form $\MU w_2.\; \const{D}( \vec u )$,
then the expansion %\rem{TODO: define}
of $\nf{\Val_1 \sigma} \ec{y}$ and $\nf{\Val_1 \sigma} \ec{z}$
are $\alpha$-disequivalent at position $p$.
Since $t_1$ and $t_2$ are minimal,
say $t_1$ is of the form $\MU w.\; \const{C}( \vec t \,)$,
and $t_2$ is ${\Varec{x}}$.
Since $\sigma$ maps ${\Varec{x}}$ to a closed $\mu$-term $\MU {\Varec{x}}.\; t$,
we have that $FV( t_1 ) \subseteq \{ {\Varec{x}} \}$,
or else the expansion of $\nf{\Val_1 \sigma} \ec{y}$ and $\nf{\Val_1 \sigma} \ec{z}$
are $\alpha$-disequivalent at position $p$ since they do not contain the same free variables.
Since $\expand{\Val_1 \ec{y}}{t_1}$ is normal,
there is a closed $\mu$-term $v \in \BAD_{\smash{\J}}^{\Varec{x}}( \Val_1 )$
such that $v \vsimv{\Varec{x}} \expand{\Val_1 \ec{y}}{t_1}$.
Thus, by assumption on the selection of $\MU {\Varec{x}}.\; t$,
we have $\MU {\Varec{x}}.\; t \not\vsimv{\Varec{x}} \expand{\Val_1 \ec{y}}{t_1}$,
which implies that the expansion of $\nf{\Val_1 \sigma} \ec{y}$ and $\nf{\Val_1 \sigma} \ec{z}$
are $\alpha$-disequivalent at position $p$. %\rem{TODO: more detail}

Thus, by induction on the number of applications of the above rule used to obtain $\ValC$,
we have that $\ValC$ satisfies (2) and (3).
\qed
\end{proof}
\end{rep}

Intuitively, this lemma states three properties of $\ValC$ that ensure a normal
interpretation $\J$ can be constructed that satisfies $\Ec$. Property~\PART{1} states that the
values in the range of $\ValC$ are $\alpha$-equivalent to a value in normal
interpretation. This means they are closed, normal, and acyclic when
required. Property~\PART{2} states that the interpretation of all subterms in the
range of $\ValC$ depends on its associated variable only. In other words, the
interpretation of a subterm $t$ where $\muvar(t) = {\Varec{x}}$ is equal to
$\ValC \ec{x}$, independently of the context. % in which it is interpreted.
Property~\PART{3} states that $\ValC$ is injective (modulo
$\alpha$-equivalence), which ensures %in the following
that distinct values are
assigned to distinct equivalence classes.

\begin{theorem}[Solution Soundness]%
\label{thm:ss}%
\afterDot
If there exists a derivation tree with root node $\Ec$ containing a saturated node, then $\Ec$ is $\thD$-satisfiable.
\end{theorem}
\begin{proof}
Let $\Fc$ be a saturated node in a derivation tree with root node $\Ec$.
%We will construct a model $\J$ for a set of equalities $\Fc$ that is equivalent to $\Fc_0$, and where
%all equivalence classes of $\Fc$ contain at least one variable, and all selectors in $\Fc$ are applied to variables only.
%The former comes with no loss of generality since new equalities of the form $y \teq t$ for fresh variable $y$ can be added to $\Fc$ without affecting its satisfiability,
%The latter also comes with no loss of generality since nested applications of selectors in $\Fc_0$ can be replaced by fresh variables while adding equalities to $\Fc\!$.
We consider a normal interpretation $\J$
that interprets wrongly applied selectors based on equality information in $\Fc$
and that interprets the variables of $\Fc$ based on the completion $\ValC$.
% of %the mapping $\Val$. % from phase~2.
For the variables, let $\interp{\J}{x\typ{\tau}}$ be the value in $\J(\tau)$ that is $\alpha$-equivalent with $\ValC \ec{x}$ for each variable $x \in \Terms(\Fc)$,
which by Lemma~\ref{lem:interpretation-completion}\PART{1} is guaranteed to exist.

We first show that $\J$ satisfies all equalities $t_1 \teq t_2 \in \Fc\!$. % where $t_1$ and $t_2$ have type $\tau$.
To achieve this, we show
by structural induction on $t\typ\tau$
that $\interp{\J}{t} \aequiv \ValC \ec{t}$ for all terms $t \in \Terms( \Fc )$,
which implies $\J \models t_1 \teq t_2$ since $\J$ is normal.

If $t$ is a variable, then $\interp{\J}{t} \vsim \ValC \ec{ t }$ by construction.

If $t$ is a constructor term of the form $\const{C}( u_1, \ldots, u_n )$,
then $\interp{\J}{t}$ is $\alpha$-equivalent with $\nf{\MU x.\; \const{C}( \interp{\J}{u_1}, \ldots, \interp{\J}{u_n} ) }$ for some fresh $x$,
which by the induction hypothesis is $\alpha$-equivalent with $\nf{\MU x.\; \const{C}( \ValC \ec{ u_1 }, \ldots, \ValC \ec{ u_n } )}$.
Call this term $t'$.
%By Lemma~\ref{lem:interpretation-completion}\PART{2} and due to our construction of $\ValC$, we have $\ValC \ec{t}$ is a term
%of the form $\MU {\Varec{t}}.\; \const{C}( w_1, \ldots, w_n )$
%where $\expand{\ValC \ec{t}}{w_i} = \ValC \ec{u_i}$ for $i \in [n]$.
Since \rn{Inject} and \rn{Clash} do not apply to $\Fc\!$,
by the construction of $\ValC$ we have that $\ValC \ec{t}$ is a term
of the form $\MU {\Varec{t}}.\; \const{C}( w_1,\allowbreak \ldots,\allowbreak w_n )$
where
$\muvar(w_i) = {\Varec{u_i}}$ for each $i$.  Thus by Lemma~\ref{lem:interpretation-completion}\PART{2},
$\expand{\ValC \ec{t}}{w_i} = \ValC \ec{u_i}$.
For each $i$, let $u_\iPrime$ be the $i$\vvthinspace th argument of $t'$.
\begin{rep}By Lemma~\ref{lem:mu-norm-arg},\end{rep}%
\begin{conf}Clearly,\end{conf} %% TODO: Fix! ;)
$\expand{t'}{u_\iPrime} \vsim \ValC \ec{u_i}$.
Thus $\expand{t'}{u_\iPrime} \vsim \expand{\ValC \ec{t}}{ w_i }$. % for each $i \in [n]$.
%By Lemma~\ref{lem:mu-cong}, we have $t' \vsim t_w$.
Thus, $\interp{\J}{t} \vsim t' \vsim \ValC \ec{t}$, and
we have $\interp{\J}{t} \vsim \ValC \ec{t}$.

If $t$ is a selector term %of the form
$\const s^k_{\negvthinspace j}( u )$,
since \rn{Split} does not apply to $\Fc\!$,
$\ec{u}$ must contain a term of the form $\smash{\const C_{\negvthinspace\jPrime}\bigl( \const s^1_{\negvthinspace\jPrime}( u ), \ldots, \const s^{n}_{\negvthinspace\jPrime}( u ) \bigr)}$ for some $\jPrime$.
%By Lemma~\ref{lem:interpretation-completion}\PART{2} and due to our construction of $\ValC$, we have $\ValC \ec{u}$ is of the form
%$\MU {\Varec{u}}.\; \const C_{\negvthinspace\jPrime}( w_1, \ldots, w_n )$,
%where $\expand{\ValC \ec{u} }{ w_i } = \ValC \ec{ \const s^i_{\negvthinspace\jPrime}( u )}$ for $i \in [n]$.
Since \rn{Inject} and \rn{Clash} are not applicable, % to $\Fc\!$,
by construction $\ValC\ec{u}$ must be of the form
$\MU {\Varec{u}}.\; \const C_{\negvthinspace\jPrime}( w_1, \ldots, w_n )$,
where
$\muvar(w_i) = \smash{\Varec{\vphantom{I}\smash{\const s^i_{\negvthinspace\jPrime}( u )}}}$
for each $i$, and thus by Lemma~\ref{lem:interpretation-completion}\PART{2},
$\expand{\ValC \ec{u}}{ w_i } = \ValC \ec{ \smash{\const s^i_{\negvthinspace\jPrime}( u )}}$.
If $j = \jPrime$, then $\interp{\J}{t}$ is $\alpha$-equivalent with $\expand{\ValC \ec{u}}{w_k}$, which is equal to $\ValC \ec{ \smash{ \const s^k_{\negvthinspace j}( u )}}
= \ValC \ec{t}$.
If $j \neq \jPrime$, since \rn{Cong} does not apply,
any term of the form $\smash{\const s^k_{\negvthinspace j}( u' )}$ not occurring in $\ec{t}$
is such that $[u] \not= [u']$.
By the induction hypothesis and Lemma~\ref{lem:interpretation-completion}\PART{3}, $\interp{\J}{u} \neq \interp{\J}{u'}$ for all such $u$, $u'$.
Thus, we may interpret $\smash{\interp{\J}{\const s^k_{\negvthinspace j}}( \interp{\J}{u} )}$ as the value in $\interp{\J}{\tau}$ that is $\alpha$-equivalent with $\ValC \ec{ t }$.

We now show that all disequalities in $\Fc$ are satisfied by $\J$.
Assume $t \tneq u \in \Fc\!$.
Since \rn{Conflict} does not apply, $t \teq u \notin \Fc$ and thus $\ec{t}$ and $\ec{u}$ are distinct.
Since $\interp{\J}{t} \vsim \ValC \ec{t}$ and $\interp{\J}{u} \vsim \ValC \ec{u}$,
by Lemma~\ref{lem:interpretation-completion}\PART{3}, $\interp{\J}{t} \neq \interp{\J}{u}$, and thus $\J \models t \tneq u$.

Since by assumption $\Fc$ contains only equalities and disequalities, we have $\J \models \Fc\!$,
and since $\Ec \subseteq \Fc\!$,
% is a superset of $\Ec$,
we conclude that $\J \models \Ec$.
\qed
\end{proof}

By Theorems~\ref{thm:t},~\ref{thm:rs}, and~\ref{thm:ss}, the
calculus is sound and complete for the universal theory of (co)datatypes. We can
rightly call it a decision procedure for that theory.
The proof of solution soundness is constructive in that it provides a
method for constructing a model for a saturated configuration, by means of the
mapping $\ValC$.

\section{Implementation as a Theory Solver in CVC4}
\label{sec:implementation-as-a-theory-solver-in-cvc4}

The decision procedure was presented at a high level of abstraction, omitting
quite a few details. This section describes the main aspects of the
implementation within the SMT solver CVC4:\ the integration into \XXXL($T$)
\cite{ganzinger-et-al-2004}, the extension to quantified formulas, and some of
the optimizations.

%%% @ANDY: I'm trying without \paragraph's, to save space but also to make the
%%% section look more like a whole than a random list of stuff. Revert if you
%%% disagree.

The decision procedure is implemented as a \relax{theory solver} of CVC4, that is, a
specialized solver for determining the satisfiability of conjunctions of
literals for its theory.
Given a theory $T = T_1 \mathrel\cup \cdots \mathrel\cup T_n$ and a set of
input clauses $F$ in conjunctive normal form, the \XXXL($T$) procedure
incrementally builds partial assignments of
truth values to the atoms of $F$ such that no clause in $F$ is falsified. We can regard such a
partial assignment as a set $M$ of true literals.
% where $a \in M$ for all atoms $a$ that
%it assigned the value $\top$ and $\neg\: a \in M$ for all atoms that it assigns
%the value $\bot$.
%at a high level,
By a variant of the Nelson--Oppen method \cite{jovanovic2011sharing,nelson-oppen-1979},
each $T_i$-solver %for $i = 1, \ldots, l$  n?
takes as input the union $M_i$
of (1)~the purified form of $T_i$-literals occurring in $M$, where fresh
variables replace terms containing symbols not belonging to~$T_i$;
(2)~additional (dis)equalities between variables of types not belonging to
$T_i$. Each $T_i$-solver
either reports that a subset $C$ of $M_i$ is $T_i$-unsatisfiable, in which case
$\neg\: C$ is added to $F\!$, adds a clause to $F\!$, or does nothing.
When $M$ is a complete assignment for $F\!$, a theory solver can choose to do
nothing only if $M_i$ is indeed $T_i$-satisfiable.

Assume $\Ec$ is initially the set $M_i$ described above. With each equality $t \teq u$ added
to $\Ec$, we associate a set of equalities from $M_i$ that together entail $t
\teq u$, which we call its \emph{explanation}.
Similarly, each $\Val \ec{x}$ is assigned an explanation---that is, a set
of equalities from $M_i$ that entail that the values of $\ec{x}$ in models of
$\Ec$ are of the form $\Val \ec{x}$. For example, if $x \teq \const{C}( x ) \in
M_i$, then $x \teq \const{C}( x )$ is an %(possible)
explanation for $\Val \ec{x} = \MU {\Varec{x}}.\; \const{C}( {\Varec{x}} )$.
The rules of the calculus are implemented as follows. For all rules with
conclusion $\bot$, we report the union of the explanations for all premises is
$\thD$-unsatisfiable. For \rn{Split}, we add the exhaustiveness clause %of the form
$\DISC{1} \mathrel\lor \cdots \mathrel\lor \DISC{m}$ to $F$.
Decisions on which branch to take are thus performed externally by the SAT
solver. % and not by the theory solver.
All other rules add equalities to the
internal state of the theory solver. The rules in phase~1 are performed
eagerly---that is, for partial satisfying assignments $M$---while the rules in
phases~2 and~3 are performed only for complete satisfying
assignments $M$.

Before constructing a model for $F\!$, the theory solver constructs neither $\mu$-terms nor the mapping $\Val$.
Instead, $\Val$ is computed implicitly by traversing the equivalence classes of $\Ec$ during phase~2.
To detect whether \rn{Acyclic} applies,
the procedure considers each equivalence class $\ec{t}$ containing a datatype constructor $\const{C}( t_1, \ldots, t_n )$.
It visits $\ec{t_1}, \ldots, \ec{t_n}$ and all constructor arguments in these equivalence classes recursively.
If while doing so it returns to $\ec{t}$, it deduces that \rn{Acyclic} applies. % to $t$.
To recognize when the precondition of \rn{Unique} holds,
the procedure considers the set $\SSS$ of all codatatype equivalence classes. It
\begin{conf}\goodbreak\noindent\end{conf}%  %%% TYPESETTING BREAK
simultaneously visits the equivalence classes of arguments of constructor terms in each equivalence class in $\SSS$,
while partitioning $\SSS$ into $\SSS_1, \ldots, \SSS_n$ based on the top-most symbol of constructor terms in these equivalence classes
and the equivalence of their arguments of ordinary types.
It then partitions each set %in this partition %in the same manner
recursively.
If %after doing so,
the resulting partition contains a set $\SSS_i$ containing distinct terms $u$
and $v$, it deduces that \rn{Unique} applies to $u$ and~$v$.

While the decision procedure is restricted to universal conjectures, in practice
we often want to solve problems that feature universal axioms and existential
conjectures.
%%% @ANDY: not sure how to phrase this now
%where checking satisfiability requires reasoning about ground literals in the
%presence of universally quantified formulas.
Many SMT solvers, including CVC4, can reason about quantified formulas using incomplete instantiation-based methods~\cite{MouraBjoerner07,ReynoldsTinelliMoura14}.
These methods extend naturally to %quantified
formulas involving datatypes and codatatypes.

However, the presence of quantified formulas poses an additional challenge in
the context of (co)datatypes. Quantified formulas may entail an upper bound on
the cardinality of an uninterpreted type $\ty{u}$.
When assuming that $\ty{u}$ has infinite cardinality, the calculus
presented in Section~\ref{sec:the-decision-procedure} is \relax{incomplete}
since it may fail to recognize cases where \rn{Split} and \rn{Single} should be
applied.
This does not %necessarily
impact the correctness of the procedure in this setting,
since %by assumption
the solver is already incomplete in the presence of quantified formulas.
Nonetheless, two techniques help increase the precision of the solver.
First, we can apply \rn{Split} to datatype terms whose cardinality depends on the finiteness of
uninterpreted types.
Second, we can conditionally apply \rn{Single} to codatatype terms that may have cardinality one.
For example, the $\ty{stream}_{\,\ty{u}}$ codatatype %, where $\ty{u}$ is uninterpreted.
has cardinality one precisely when $\ty{u}$ has cardinality one.
If there exist two equivalence classes $\ec{s}$ and $\ec{t}$ for this type,
the implementation adds the clause %of the form
$(\exists x\;y\typ{\ty{u}}.\;\, x \not\teq y ) \mathrel\vee s \teq t$ to $F\!$.
\begin{rep}In other words, either the cardinality of $\ty{u}$ is greater than one or $s$ is equal to $t$.\end{rep}

The implementation of the decision procedure uses several optimizations
following the lines of Barrett et al.\ \cite{barrett-et-al-2007}.
%and are not reflected in the presentation of the calculus.
\begin{rep}We briefly mention the main ones. \end{rep}%
Discriminators are part of the signature and not abbreviations.
This requires extending the decision procedure with several
rules, % \cite{barrett-et-al-2007},
which apply uniformly to datatypes and codatatypes.
This approach often leads to better performance because it introduces terms
less eagerly to $\tEc$.
Selectors are collapsed eagerly:
If $\const s^k_{\negvthinspace j}( t ) \in \tEc$ and $t =
\const{C}_{\negvthinspace j}( u_1, \ldots, u_n )$, the solver directly adds
$\smash{\const s^k_{\negvthinspace j}( t ) \teq u_k}$ to $\Ec$, whereas the presented calculus
would apply \rn{Split} and \rn{Inject} before adding this equality.
To reduce the number of unique literals considered by the calculus, we
compute a normal form for literals as a preprocessing step. In particular, we
replace $u \teq t$ by $t \teq u$ if $t$ is smaller than $u$ with respect to some term
ordering, replace $\smash{\const{C}_{\negvthinspace j}( \vec t \,) \teq
\const{C}_{\negvthinspace\jPrime}( \vec u )}$ with $\bot$ when $j \neq \jPrime$,
replace all selector terms of the form $\smash{\const{s}^{\,k}_{\negvthinspace j}(
\const{C}_{\negvthinspace j}(t_1,\ldots,t_n) )}$ by $t_k$, and replace
occurrences of discriminators $\smash{\const{d}_{\negvthinspace j}(
\const{C}_{\negvthinspace\jPrime}( \vec t \,) )}$ by $\top$ or $\bot$
based on whether $j = \jPrime$.

% AJR : this paragraph may have been what confused Reviewer #2. It does suggest that we are copying verbatim what was in CVC3 (which is what Reviewer #2 thought).
% Since we aren't saying anything new beyond what is said in Barrett et al (and honestly, we didn't critically evaluate the performance of the updated implementation), 
% maybe we should just remove this paragraph.

As Barrett et al.\ \cite{barrett-et-al-2007} observed for their procedure,
it is both theoretically and
empirically beneficial to delay applications of \rn{Split} as long as
possible. Similarly, \rn{Acyclic} and \rn{Unique} are fairly expensive because
they require traversing the equivalence classes, which is why they are part of
phase~2.

\kern-1pt

%\section{Examples}
%\label{sec:examples}

\section{Evaluation on Isabelle Problems}
\label{sec:experimental-results}

\newcommand\gandl{G\&L}
\newcommand\HD[1]{\hbox to2.25em{\footnotesize\hfill\!\!\!\!#1\!\!\!\!\hfill}}

To evaluate the decision procedure, we generated benchmark problems from existing
Isabelle formalizations using Sledgehammer \cite{blanchette-et-al-2013-smt}.
%There are not yet many benchmarks available, since codatatypes are a
%recent addition to Isabelle \cite{blanchette-et-al-2014-impl}. \rem{Necessary to mention this? Seems to detract from significance of contribution.}
We
included all the theory files from the Isabelle distribution (Distro, 879~goals)
and the \emph{Archive of Formal Proofs} (AFP, 2974~goals) \cite{klein-et-al-afp}
that define codatatypes falling within the supported fragment. We
added two unpublished theories about Bird and Stern--Brocot trees
by Peter Gammie and Andreas Lochbihler (\gandl,
317~goals). To exercise the datatype support, theories about lists and trees
were added to the first two benchmark sets. The theories were selected before
conducting any experiments. The experimental data are available
online.%
\footnote{\url{http://lara.epfl.ch/~reynolds/CADE2015-cdt/}}

%http://www21.in.tum.de/~blanchet/codp_data.tgz}}

%  & 879 & 2974\phantom{0} & 317 & 4170

For each goal in each theory, Sledgehammer selected about 16~lemmas,
%\end{conf}%
%\begin{rep}a fixed number of lemmas\end{rep},
which were monomorphized and translated to SMT-LIB~2 along with the goal.
The resulting problem was given to the
development version of CVC4 and to Z3 4.3.2 for comparison, each running for
up to 60~s on the StarExec cluster
\cite{stump-et-al-2014-starexec}.
Problems not involving any (co)datatypes were left out.
Due to the lack of machinery for reconstructing inferences about (co)datatypes
in Isabelle, the solvers are trusted as oracles. % in these experiments.
%Z3 is included for comparison.

The development version of CVC4 was run on each problem several times, with the
support for datatypes and codatatypes either enabled or disabled. The
contributions of the acyclicity and uniqueness rules were also measured.
% to find
% out whether these somewhat expensive rules are useful in practice.
Even when the
decision procedure is disabled, the problems may contain
basic lemmas about constructors and selectors, allowing some (co)datatype
reasoning.
% when this makes sense.
%This allows us to answer the question, \relax{What are the
%benefits of activating the decision procedure as opposed to letting
%Sledgehammer do what it would normally do?}

%  * experience with such features is not extremely high -- e.g. arithmetic,
%    a most useful theory on lots of benchmarks, increases Sledgehammer's
%    success rate by 2 percentage points, or 4\%, with Z3 in earlier work \cite{xxx}

%  * so when analyzing statistics of this kind, with highly varied problems,
%    we must not overestimate the importance of a single trick
%  * on the other hand: every percentage point counts! cite Tom Hales

\newcommand\nummodden{num\hbox{\rm\_}mod\hbox{\rm\_}den}

\newcommand\captioni{\,Number of solved goals for the three benchmark suites}
%\begin{rep}
%\renewcommand\captioni{\,Number of solved goals with 16~lemmas per goal}
%\end{rep}

\begin{table*}[t!]
\normalsize
\begin{center}\begin{tabular}{@{\;}l@{\kern1.5em}cc@{\kern.5em}cc@{\kern.5em}cc@{\kern1.3em}cc@{\;}}
  & \multicolumn{2}{c}{Distro} & \multicolumn{2}{c}{AFP} & \multicolumn{2}{c}{\gandl} & \multicolumn{2}{c}{Overall} \\[-0.5pt]
  & \HD{CVC4} & \HD{Z3} & \HD{CVC4} & \HD{Z3} & \HD{CVC4} & \HD{Z3} & \HD{CVC4}& \HD{Z3} 
\MIDRULE
No (co)datatypes
  & 221 & 209 & 775 & 777 & 52 & 51 & 1048 & 1037 \\%[\jot]
%Datatype simplification only
%  &  \\
Datatypes without \rn{Acyclic}
  & 227 & -- & 780 & -- & 52 & -- & 1059 & -- \\
Full datatypes
  & 227 & 213 & 786 & 791 & 52 & 51 & 1065 & 1055 \\%[\jot]
%Codatatype simplification only
%  &  \\
Codatatypes without \rn{Unique}\!\!
  & 222 & -- & 804 & -- & 56 & -- & 1082 & -- \\
Full codatatypes
  & 223 & -- & 804 & -- & \bfseries 59 & -- & 1086 & -- \\%[\jot]
Full (co)datatypes
  & \bfseries 229 & -- & \bfseries 815 & -- & \bfseries 59 & -- & \bfseries 1103 & -- %\\[\jot]
\end{tabular}\end{center}
\caption{\captioni}
\label{tab:benchsixteen}
\end{table*}

The results are summarized in Table~\ref{tab:benchsixteen}.
%The goals in Table \ref{tab:benchsixteen} include 16
%heuristically selected lemmas (before monomorphization);
%\cite{meng-paulson-2009-relev};
%that number is increased to 256 for Table \ref{tab:benchtwofiftysix}.
The decision procedure
makes a difference across all three benchmark suites. It accounts for an overall
success rate increase of over \OK{5\%}, which is quite significant.
%For Isabelle
%users, the practical usefulness of interpreted (co)datatypes is roughly
%comparable to that of arithmetic \cite{blanchette-et-al-2013-smt}, although this
%obviously depends on the application.
The raw evaluation data also
suggest that the theoretically stronger decision procedures almost completely
subsume the weaker ones in practice: We encountered only one goal (out of \OK{4170})
that was solved by a configuration of CVC4 and unsolved by a configuration of
CVC4 with more features enabled.

Moreover, every aspect of the procedure, including the
more expensive rules, make a contribution.
%* and finally, we look in more detail at one or two such proofs, by presenting
%  it and explaining it (and, before that, understanding it)
%
Three proofs were found thanks to
the acyclicity rule and four required uniqueness. Among the latter,
three are simple proofs of the form
\keyw{by}~\textit{coinduction}~\textit{auto}
in Isabelle\begin{rep} \cite{blanchette-et-al-2014-impl}\end{rep}.
The fourth proof, by Gammie and Lochbihler, is more elaborate:
%
\begin{quote}
\keyw{lemma} \,\textit{\nummodden\_unique}: \,$x = \const{Node}\;0\;\const{num}\; x \vthinspace\Longrightarrow\vthinspace x = \const{\nummodden}$ \\
\keyw{proof} \,(\textit{coinduction arbitrary}: $x$ \textit{rule}: \textit{tree.coinduct\_strong}) \\
\noindent\hbox{}\quad  \keyw{case} (\textit{Eq\_tree} $x$) \keyw{show} \textit{?case} \\
\noindent\hbox{}\qquad  \keyw{by} (\textit{subst} (1 2 3 4) \textit{Eq\_tree}) (\textit{simp add}: \textit{eqTrueI}[OF \textit{Eq\_tree}]) \\
\keyw{qed}
\end{quote}
%
\noindent
where \const{\nummodden} is defined as $\const{\nummodden} =
\const{Node}\;0\;\const{num}\;\const{\nummodden}$.

\section{Conclusion}
\label{sec:conclusion}

We introduced a decision procedure for the universal theory of datatypes and
codatatypes. Our main
contribution has been the support for codatatypes. Both the metatheory and
the implementation in CVC4 rely on $\mu$-terms to represent cyclic values.
Although this aspect is primarily motivated by codatatypes, it enables
a uniform account of datatypes and codatatypes---in particular,
the acyclicity rule for datatypes exploits $\mu$-terms to detect cycles.
% are not very
% difficult per se, but there are tricky corner cases to take care of if we care
% about completeness (which we do).
%% alliteration
The empirical results on Isabelle benchmarks
confirm that CVC4's new capabilities improve the state of the art.

This work is part of a wider program that aims at enriching automatic provers
with high-level features and at reducing the gap between automatic and
interactive theorem proving. As future work, it would be useful to implement
proof reconstruction for (co)data\-type inferences in Isabelle.
CVC4's finite model finding capabilities \cite{reynolds-et-al-2013} could also
be interfaced for generating counterexamples in proof assistants;
%as an alternative to Nitpick;
in this context, the acyclicity and uniqueness rules are
crucial to exclude spurious countermodels. Finally, it might be worthwhile to
extend SMT solvers with dedicated support for (co)recursion.

%Finally, it might be possible to go
%further in terms of supporting nested and mixed (co)recursion and quantified
%ormulas over (co)datatypes in solvers.

{\footnotesize
\def\ackname{Acknowledgment}
\paragraph{\footnotesize\ackname.}
\looseness=-1
We owe a great debt to the development team of CVC4,
including Clark Barrett and Cesare Tinelli,
and in particular Morgan Deters,
who jointly with the first author
developed the initial version
of the theory solver for datatypes in CVC4.
%
Our present and former bosses, Viktor Kuncak, Stephan Merz, Tobias Nipkow,
Cesare Tinelli, and Christoph Weidenbach, have either encouraged the research on
codatatypes or at least benevolently tolerated it, both of which we are thankful
for.
%
Peter Gammie and Andreas Lochbihler provided useful benchmarks.
Andrei Popescu helped clarify our thoughts regarding codatatypes and indicated
related work. Dmitriy Traytel took part in discussions about degenerate
codatatypes.
%
Pascal Fontaine, Andreas Lochbihler, Andrei Popescu, Christophe Ringeissen, Mark
Summerfield, Dmitriy Traytel, and the anonymous reviewers suggested many
textual improvements. The second author's research was partially supported by
the Deutsche
Forschungs\-gemein\-schaft %(DFG)
project ``Den Hammer h\"arten'' (grant NI\,491\slash 14-1)
and the Inria technological development
action ``Contre-exemples Utilisables par Isabelle et Coq'' (CUIC).

}

\bibliographystyle{splncs03}
\bibliography{bib}{}

\end{document}
